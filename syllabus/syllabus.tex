\documentclass[12pt]{article}
\usepackage{alltt}
\usepackage{color}
\usepackage{hyperref}
\usepackage[margin=1in]{geometry}
\setlength{\textwidth}{7in}
\setlength{\textheight}{9in}
\setlength{\topmargin}{-0.5in}

\newcommand{\li}{\item}
\newcommand{\myitem}[1]{\item[#1]}

\begin{document}

\begin{center}
{\Large Syllabus, CSCI297b, Scientific Visualization, Winter 2023}
\end{center}

\begin{description}

\myitem{Instructor:}
Dr. Geoffrey Matthews,
Parmly 407A, {\tt gmatthews <at>  wlu <dot> edu}

\myitem{Web page:} \url{https://github.com/geofmatthews/csci312} 

\myitem{Office hours:} MTWRF 9:30 and by appointment

\myitem{Lectures:} MTWRF 10:30-12:30, Parmly 405

\myitem{Goals:} This course is an introduction to graphing scientific data.
We will study the fundamentals of preparing data, graphing it, and 
presenting these graphcs on the web.  We will also study design
and human engineering principles to make our graphs more
effective.  By the end of the course the student should be able to:
\begin{itemize}
\li Tidy up datasets from a number of sources so that they are easy
to graph in a variety of ways.
\li Understand various graph techniques and be able to justify 
their use in various situations.
\li Understand principles of communication that make some graphs
good communicators, and other graphs confusing, misleading, or
wrong.
\li Be able to apply these principles to create new graphs
in a wide variety of contexts using R and RStudio.
\end{itemize}

\myitem{Texts:}~

\begin{itemize}

\li
\url{https://alexd106.github.io/intro2R/index.html}
\dotfill
An Introduction to R

\li
\url{https://r4ds.had.co.nz/}
\dotfill
R for Data Science


\li
\url{https://clauswilke.com/dataviz/}
\dotfill
Fundamentals of Data Visualization
\end{itemize}

\myitem{Software:} ~

We will primarily be using the RStudio Workbench server, provided
by the university at \url{https://rstudioworkbench.wlu.edu/}.

However, if you want to install R and RStudio on your own
computer, they are available here:
\begin{itemize}
\li
\url{https://www.r-project.org/} \dotfill R
\li
\url{https://posit.co/download/rstudio-desktop/}\dotfill RStudio
\end{itemize}


\myitem{Grading:}  Grading will be based on:
\begin{description}

\item[Class exercises:]   Most days class lab exercises will be 
interspersed with lectures on R and visualizations.  If you must miss a day,
you are responsible for making up the work on your own time.  20\%

\item[Graph of the day:] Every day one or two students will present
a short (5 minute) report on a particularly good or bad graph found on
the internet.   We will prepare a schedule of dates for students the
first day.  Each student should prepare at least one day in advance
so that if someone misses their day the next person in line can step up.
 20\% 

\item[Report:]  A report on a dataset 
written in Rmarkdown submitted the last day of class.  20\%

\item[Oral presentation:] A brief (10-15 minute) oral presentation
about the dataset analyzed in your written report.  20\%

\item[Final exam:]  The final exam is comprehensive. It will be an
online multiple choice exam.  It is closed book and closed notes.
You may not consult with other people or any other resources
during the exam.  20\%

\end{description}
\item[Letter grades:]
A $\ge$ 90\% $>$ B $\ge$ 80\% $>$ C $\ge$ 70\% $>$ D $\ge$ 60\% $>$ F

\item[Daily routine:]~

\begin{tabular}{ll}
Time & Activity \\\hline
10:30  & Lecture  on R and in-class exercises \\
11:40  & Break \\
11:50  & Graph of the Day \\
11:55  & Lecture \& Discussion on Fundamentals\\
12:30 & Class ends
\end{tabular}

\myitem{Schedule:}~

\begin{tabular}{llccc}
&&\multicolumn{2}{c}{\hrulefill Lab\hrulefill} & Principles \\
Date & Day & \href{https://intro2r.com/}{intro2r}
      & \href{https://r4ds.hadley.nz/}{RfDS2e}
       & \href{https://clauswilke.com/dataviz/index.html}{FoDV} \\\hline
2023-04-24     &     Monday     & 1 &  &    \\
 2023-04-25     &        Tuesday  & 2 & & 1,2,3      \\
 2023-04-26    &       Wednesday  & 2 & & 4,5      \\
 2023-04-27    &        Thursday     & 3 & & 6,7    \\
 2023-04-28    &          Friday      & 3 & &  8    \\\hline
 2023-05-01    &          Monday   & 4&   & 9,10   \\
 2023-05-02    &         Tuesday    & 4 & & 11,12    \\
 2023-05-03    &       Wednesday & 5 &  &  13,14    \\
 2023-05-04    &        Thursday   & 5 &   &  15   \\
 2023-05-05    &          Friday      &  8&  & 16    \\\hline
 2023-05-08    &          Monday   & 8 &  & 17   \\
 2023-05-09    &         Tuesday    &  & 10 & 18,19    \\
 2023-05-10    &       Wednesday &   &  10  & 20,21    \\
 2023-05-11   &         Thursday&  &  11&  22    \\
 2023-05-12   &           Friday   &   & 11  &  23,24,25  \\\hline
 2023-05-15   &           Monday&   & 12   &   26,27    \\
 2023-05-16   &          Tuesday&   &   12  &   28,29   \\
 2023-05-17   &        Wednesday&   \multicolumn{3}{c}{Student presentations}       \\
 2023-05-18   &         Thursday & \multicolumn{3}{c}{Student presentations}       \\
 2023-05-19   &           Friday   &  \multicolumn{3}{c}{Student presentations}     \\\hline
 \multicolumn{4}{l}{Final Exam}
\end{tabular}

\end{description}
\end{document}

\documentclass{beamer}
\usetheme{Singapore}
\usepackage{changepage}

%\usepackage{pstricks,pst-node,pst-tree}
\usepackage{amssymb,latexsym}
\usepackage{tikz}
\usepackage{graphicx}
\usepackage{fancyvrb}
\usepackage{hyperref}
\usepackage{fancybox}
\usepackage[listings]{tcolorbox}

\definecolor{codegreen}{rgb}{0,0.6,0}
\definecolor{codegray}{rgb}{0.5,0.5,0.5}
\definecolor{codepurple}{rgb}{0.58,0,0.82}
\definecolor{backcolour}{rgb}{0.95,0.95,0.92}

\lstdefinestyle{mystyle}{
    language=Python,
    backgroundcolor=\color{backcolour},   
    commentstyle=\color{codegreen},
    keywordstyle=\color{magenta},
    numberstyle=\tiny\color{codegray},
    stringstyle=\color{codepurple},
    basicstyle=\ttfamily\scriptsize,
    breakatwhitespace=false,         
    breaklines=true,                 
    captionpos=b,                    
    keepspaces=true,                 
    numbers=left,                    
    numbersep=5pt,                  
    showspaces=false,                
    showstringspaces=false,
    showtabs=false,                  
    tabsize=2,
    escapechar=|,
    frame=single
}

\lstset{style=mystyle}


\newcommand{\bi}{\begin{itemize}}
\newcommand{\li}{\item}
\newcommand{\ei}{\end{itemize}}
\newcommand{\Show}[1]{
\begin{center}
\shadowbox{\begin{minipage}{0.8\textwidth}
          #1
          \end{minipage}}
\end{center}
}
\newcommand{\arrow}{\ensuremath{\rightarrow}}

\newcommand{\uparr}{\ensuremath{\uparrow}}


\newcommand{\fig}[2]{\centerline{\includegraphics[width=#1\textwidth]{#2}}}

\newcommand{\figg}[2]{\includegraphics[width=#1\textwidth]{#2}}

\newcommand{\bfr}[1]{\begin{frame}[fragile]\frametitle{{ #1 }}}
\newcommand{\efr}{\end{frame}}

\newcommand{\cola}[1]{\begin{columns}\begin{column}{#1\textwidth}}
\newcommand{\colb}[1]{\end{column}\begin{column}{#1\textwidth}}
\newcommand{\colc}{\end{column}\end{columns}}


\author{Fundamentals of Data Visualization}
\title{Chapter 12}

\begin{document}

\begin{frame}
\maketitle
\end{frame}

\bfr{Associations among two or more quantitative variables}
\bi
\li height, weight, length, daily energy demands
\li pH, alkalinity, nitrate/nitrite
\ei

\end{frame}
\bfr{Associations among two or more quantitative variables}
\bi
\li scatter plot
\li bubble plot
\li scatter plot matrix
\li correlogram
\li dimensionality reduction
\bi\li principal components\ei

\ei

\end{frame}

\bfr{Bluejays dataset}
\begin{verbatim}
> library(Stat2Data)

> glimpse(BlueJays)
Rows: 123
Columns: 9
$ BirdID     <fct> 0000-00000, 1142-05901, 114…
$ KnownSex   <fct> M, M, M, F, M, F, M, M, F, …
$ BillDepth  <dbl> 8.26, 8.54, 8.39, 7.78, 8.7…
$ BillWidth  <dbl> 9.21, 8.76, 8.78, 9.30, 9.8…
$ BillLength <dbl> 25.92, 24.99, 26.07, 23.48,…
$ Head       <dbl> 56.58, 56.36, 57.32, 53.77,…
$ Mass       <dbl> 73.30, 75.10, 70.25, 65.50,…
$ Skull      <dbl> 30.66, 31.38, 31.25, 30.29,…
$ Sex        <int> 1, 1, 1, 0, 1, 0, 1, 1, 0, …
> 
\end{verbatim}
\end{frame}
\bfr{Bluejays scatterplot}
\fig{.8}{blue-jays-scatter-1.png}
\end{frame}
\bfr{John Snow's 1854 cholera map}
\fig{.9}{snow-cholera-map.jpg}
\end{frame}
\bfr{Bluejays scatterplot by sex}
\fig{.8}{blue-jays-scatter-sex-1.png}
\end{frame}
\bfr{Adding skull size with bubbles}
\fig{1}{blue-jays-scatter-bubbles-1.png}
\bi
\li Head length is from tip of bill to back of head.
\li Skull size does not include the bill.
\li Plotting a fourth variable with bubble size
\ei
\end{frame}

\bfr{Bubble charts}
\bi
\li Bubble charts show quantitative variables with two aesthetics:%
\bi\li position \li size\ei
\li Difficult to visualize the strength of association.
\li Bubble size is harder to perceive than position
\ei
\end{frame}

\bfr{All-against-all scatterplot matrix}
\fig{1}{blue-jays-scatter-all-1.png}
\end{frame}

\bfr{Correlation coefficients}
\fig{1}{correlations-1.png}
\end{frame}

\bfr{Correlation coefficients}
\[
r = \frac{\sum_i (x_i - x)(y_i - y)}{\sqrt{\sum_i(x_i-x)^2}\sqrt{\sum_i(y_i-y)^2}}
\]
where $x_i$ and $y_i$ are two sets of observations and $x$ and $y$
are the sample means.
\bi
\li Symmetric in $x$ and $y$
\li Only depend on the differences from the mean, so independent of shifting.
\li Independent of scaling, too, since a constant $C$ will appear in both
numerator and denominator
\ei
\end{frame}

\bfr{Root mean square}
Correlation coefficient:
\[
r = \frac{\sum_i (x_i - x)(y_i - y)}{\sqrt{\sum_i(x_i-x)^2}\sqrt{\sum_i(y_i-y)^2}}
\]
Root mean square:
\[
\sqrt{\frac{\sum_i a_i^2}{n}}
\]
Standard deviation:
\[
\sqrt{\frac{\sum_i(x_i-x)^2}{n}}
\]
\end{frame}


\bfr{Correlegram maps the correlations to a visualization}
\fig{.6}{forensic-correlations1-1.png}
\bi\li Forensic data on glass samples\ei
\end{frame}

\bfr{Size and color together visualize small values better}
\fig{.6}{forensic-correlations2-1.png}
\end{frame}


\begin{frame}
\fig{.8}{unnamed-chunk-10-1.png}
\vfill
\tiny
\url{https://janhove.github.io/teaching/2016/11/21/what-correlations-look-like}
\end{frame}

\bfr{Principal components}
\url{https://builtin.com/data-science/step-step-explanation-principal-component-analysis}
\end{frame}

\bfr{Dimensionality reduction}
\bi
\li Many variables are correlated:
\bi
\li height
\li weight
\li arm length, leg length
\li chest, waist, leg circumference
\ei
\li If we combine these into a single super-variable
we may be able to see other, more subtle factors.
\ei
\end{frame}

\bfr{Principal components}
\fig{1}{blue-jays-PCA-1.png}
\bi
\li Scale each variable to zero mean and unit variance
\li A rigid rotation of the data around the origin
\li Each component has zero correlation with the others
\li The first component has the most variance, the second
the second most, {\em etc.}
\ei
\end{frame}

\bfr{Correlation with normalized variables}
\begin{align*}
r &= \frac{\sum_i (x_i - x)(y_i - y)}{\sqrt{\sum_i(x_i-x)^2}\sqrt{\sum_i(y_i-y)^2}}\\
&= \frac{\sum_i (x_i )(y_i )}{\sqrt{\sum_i(x_i)^2}\sqrt{\sum_i(y_i)^2}}\\
&= \frac{\sum_i (x_i )(y_i )}{1\cdot 1}\\
&= {\sum_i (x_i )(y_i )}\\
&= \vec{x} \cdot \vec{y}\\
&= \vec x^T \vec y\\
\end{align*}
\end{frame}

\bfr{Principal components}
\fig{1}{blue-jays-PCA-1.png}
\bi
\li Usually key features of the data can be seen in
the first two or three components.
\li Two things to look at:
\bi
\li the composition of the components
\li the positions of the points on the components
\ei
\ei
\end{frame}

\bfr{Composition of the components}
\fig{.65}{forensic-PCA-rotation-1.png}
\bi
\li The components are linear combinations of the variables
\ei
\end{frame}

\bfr{Project points onto the components}
\fig{1}{forensic-PCA-1.png}
\end{frame}

\bfr{Principal components}
\bi
\li Sometimes the dominant trend is known:
\bi
\li Summer-winter
\li Time
\li Treatment 
\ei
\li Verify that PC1 captures this trend.
\li Look for other trends in PC2, PC3, etc.
\ei
\end{frame}

\bfr{Paired data}
\bi
\li Data where there are two or more measurements of the same quantity under slightly different conditions.
\li Examples include:
\bi
\li two comparable measurements on each subject (e.g., the length of the right and the left arm of a person), 
\li repeat measurements on the same subject at different time points (e.g., a person’s weight at two different times during the year),
\li or measurements on two closely related subjects (e.g., the heights of two identical twins). 
\ei
\ei
\end{frame}

\bfr{Scatter plot with diagonal line where $x=y$}
\fig{.7}{CO2-paired-scatter-1.png}
\bi
\li Countries are consistent over the decades
\li Systematic shift to higher emissions
\ei
\end{frame}

\bfr{Slopegraph, for smaller numbers of subjects}
\fig{.7}{CO2-slopegraph-1.png}
\end{frame}

\bfr{Slopegraph can be used for multiple years}
\fig{.7}{CO2-slopegraph-three-year-1.png}
\end{frame}

\end{document}

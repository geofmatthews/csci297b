\documentclass{beamer}
\usetheme{Singapore}
\usepackage{changepage}

%\usepackage{pstricks,pst-node,pst-tree}
\usepackage{amssymb,latexsym}
\usepackage{tikz}
\usepackage{graphicx}
\usepackage{fancyvrb}
\usepackage{hyperref}
\usepackage{fancybox}

\newcommand{\bi}{\begin{itemize}}
\newcommand{\li}{\item}
\newcommand{\ei}{\end{itemize}}
\newcommand{\Show}[1]{
\begin{center}
\shadowbox{\begin{minipage}{0.8\textwidth}
          #1
          \end{minipage}}
\end{center}
}
\newcommand{\arrow}{\ensuremath{\rightarrow}}

\newcommand{\uparr}{\ensuremath{\uparrow}}


\newcommand{\fig}[2]{\centerline{\includegraphics[width=#1\textwidth]{#2}}}

\newcommand{\figg}[2]{\includegraphics[width=#1\textwidth]{#2}}

\newcommand{\bfr}[1]{\begin{frame}[fragile]\frametitle{{ #1 }}}
\newcommand{\efr}{\end{frame}}

\newcommand{\cola}[1]{\begin{columns}\begin{column}{#1\textwidth}}
\newcommand{\colb}[1]{\end{column}\begin{column}{#1\textwidth}}
\newcommand{\colc}{\end{column}\end{columns}}


\author{Fundamentals of Data Visualization}
\title{Chapter 14}

\begin{document}

\begin{frame}
\maketitle
\end{frame}

\bfr{Visualizing Trends}
\bi
\li When making scatter plots  or time series, we are often more interested in the overarching trend of the data than in the specific detail. 
\li By drawing the trend  we can create a visualization that helps the reader immediately see key features of the data.
\li There are two fundamental approaches to determining a trend: 
\bi
\li We can  smooth the data by some method.
\li We can fit a curve with a defined functional form. 
\ei
\li Once we have identified a trend we can look specifically at deviations from the trend.
\li Or we can separate the data into multiple components, including the underlying trend, any existing cyclical components, and episodic components or random noise.
\ei
\end{frame}

\bfr{Dow Jones for 2009}
\fig{1}{dow-jones-1.png}
\bi
\li How can we show the dominant trends without the noise?
\ei
\end{frame}

\bfr{Smoothing}
\bi
\li To generate a moving average, we take a time window, say the first 20 days in the time series.
\li Calculate the average price over these 20 days. 
\li Then move the time window by one day, so it now spans the 2nd to 21st day.
\li Calculate the average over these 20 days.
\li Move the time window again, and so on.
\li The result is a new time series consisting of a sequence of averaged prices.
\ei
\end{frame}


\begin{frame}
\cola{0.3}
\bi
\li Financial analysts usually plot the smooth curve at the end point.
\li Statisticians usually plot the smooth curve at the center of the window.
\ei
\colb{0.7}
\fig{1.1}{dow-jones-moving-ave-1.png}
\colc
\end{frame}
\end{document}

\documentclass{beamer}
\usetheme{Singapore}
\usepackage{changepage}

%\usepackage{pstricks,pst-node,pst-tree}
\usepackage{amssymb,latexsym}
\usepackage{tikz}
\usepackage{graphicx}
\usepackage{fancyvrb}
\usepackage{hyperref}
\usepackage{fancybox}

\newcommand{\bi}{\begin{itemize}}
\newcommand{\li}{\item}
\newcommand{\ei}{\end{itemize}}
\newcommand{\Show}[1]{
\begin{center}
\shadowbox{\begin{minipage}{0.8\textwidth}
          \bf #1
          \end{minipage}}
\end{center}
}
\newcommand{\arrow}{\ensuremath{\rightarrow}}

\newcommand{\uparr}{\ensuremath{\uparrow}}


\newcommand{\fig}[2]{\centerline{\includegraphics[width=#1\textwidth]{#2}}}

\newcommand{\figg}[2]{\includegraphics[width=#1\textwidth]{#2}}

\newcommand{\bfr}[1]{\begin{frame}[fragile]\frametitle{{ #1 }}}
\newcommand{\efr}{\end{frame}}

\newcommand{\cola}[1]{\begin{columns}\begin{column}{#1\textwidth}}
\newcommand{\colb}[1]{\end{column}\begin{column}{#1\textwidth}}
\newcommand{\colc}{\end{column}\end{columns}}


\author{Fundamentals of Data Visualization}
\title{Chapter 17\\The principle of proportional ink}

\begin{document}

\begin{frame}
\maketitle
\end{frame}

\bfr{The principle of proportional ink}

\begin{center}
\Show{The sizes of shaded areas in a visualization need to be proportional to the data values they represent.}
\end{center}

\end{frame}

\bfr{Bad boxplot}
\fig{1}{hawaii-income-bars-bad-1.png}
\end{frame}

\bfr{Good boxplot}
\fig{1}{hawaii-income-bars-good-1.png}
\end{frame}

\bfr{Bad lineplot}
\fig{1}{fb-stock-drop-bad-1.png}
\end{frame}

\bfr{Good lineplot}
\fig{1}{fb-stock-drop-good-1.png}
\end{frame}

\bfr{Bars for change}
\fig{1}{hawaii-income-change-1.png}
\end{frame}

\bfr{Bars for change}
\fig{1}{fb-stock-drop-inverse-1.png}
\centerline{Loss of Facebook stock price}
\end{frame}

\bfr{Log plots of amounts}
\fig{1}{oceania-gdp-logbars-1.png}
\bi
\li No natural zero.
\ei
\end{frame}

\bfr{Log plots of amounts}
\fig{1}{oceania-gdp-logbars-long-1.png}
\bi
\li No natural zero.
\ei
\end{frame}

\bfr{Log plots of amounts: use dots}
\fig{1}{oceania-gdp-dots-1.png}
\end{frame}

\bfr{Log plots of ratios}
\fig{1}{oceania-gdp-relative-1.png}
\bi
\li Start at 1, not zero.
\ei
\end{frame}

\bfr{Direct area visualizations}
\fig{1}{RI-pop-pie-1.png}
\centerline{Number of inhabitants in Rhode Island counties}
\end{frame}

\bfr{Direct area visualizations}
\fig{1}{RI-pop-bars-1.png}
\centerline{Number of inhabitants in Rhode Island counties}
\end{frame}

\bfr{Direct area visualizations}
\fig{1}{RI-pop-treemap-1.png}
\centerline{Number of inhabitants in Rhode Island counties}
\end{frame}

\end{document}

\documentclass{beamer}
\usetheme{Singapore}
\usepackage{changepage}

%\usepackage{pstricks,pst-node,pst-tree}
\usepackage{amssymb,latexsym}
\usepackage{tikz}
\usepackage{graphicx}
\usepackage{fancyvrb}
\usepackage{hyperref}
\usepackage{fancybox}

\newcommand{\bi}{\begin{itemize}}
\newcommand{\li}{\item}
\newcommand{\ei}{\end{itemize}}
\newcommand{\Show}[1]{
\begin{center}
\shadowbox{\begin{minipage}{0.8\textwidth}
          #1
          \end{minipage}}
\end{center}
}
\newcommand{\arrow}{\ensuremath{\rightarrow}}

\newcommand{\uparr}{\ensuremath{\uparrow}}


\newcommand{\fig}[2]{\centerline{\includegraphics[width=#1\textwidth]{#2}}}

\newcommand{\figg}[2]{\includegraphics[width=#1\textwidth]{#2}}

\newcommand{\bfr}[1]{\begin{frame}[fragile]\frametitle{{ #1 }}}
\newcommand{\efr}{\end{frame}}

\newcommand{\cola}[1]{\begin{columns}\begin{column}{#1\textwidth}}
\newcommand{\colb}[1]{\end{column}\begin{column}{#1\textwidth}}
\newcommand{\colc}{\end{column}\end{columns}}


\author{Fundamentals of Data Visualization}
\title{Chapter 13}

\begin{document}

\begin{frame}
\maketitle
\end{frame}

\bfr{Time series}
\bi
\li Time (and a few other variables) imposes additional structure on the data
\li  The data points have an inherent order; we can arrange the points in order of increasing time and define a predecessor and successor for each data point.
\li We use {\bf line graphs} to connect point in order.
\li Can be used whenever one variable imposes a total order.
\li A treatment can be purposefully set to a range of values, e.g.

\ei
\end{frame}

\bfr{Monthly preprint submissions to bioRxiv, scatterplot}
\fig{1}{biorxiv-dots-1.png}
\end{frame}

\bfr{Monthly preprint submissions to bioRxiv, line graph}
\fig{1}{biorxiv-dots-line-1.png}
\end{frame}

\bfr{Line plots}
\bi
\li Since each point has unique left and right neighbors, lines can connect these.
\li Some  object to drawing lines between points because the lines do not represent observed data.
\li Had observations been made at intermediate times they would probably not have fallen exactly onto the lines shown.
\li Tthe lines correspond to made-up data. 
\li Yet they may help with perception when the points are spaced far apart or are unevenly spaced.
\li Using lines to represent time series is generally accepted practice
\ei
\end{frame}

\bfr{Omit the dots}
\fig{1}{biorxiv-line-1.png}
\end{frame}

\bfr{Fill with color}
\fig{1}{biorxiv-line-area-1.png}
\end{frame}

\bfr{Multiple time series need connecting lines}
\fig{1}{bio-preprints-dots-1.png}
\end{frame}

\bfr{Multiple time series need connecting lines}
\fig{1}{bio-preprints-lines-1.png}
\end{frame}

\bfr{Omit dots and legends when you can}
\fig{1}{bio-preprints-direct-label-1.png}
\end{frame}

\bfr{Dose-response curves}
\fig{1}{oats-yield-1.png}
\end{frame}

\bfr{Two time series compared}
\fig{1}{house-price-unemploy-1.png}
\end{frame}

\bfr{Connected scatter plot}
\fig{.8}{house-price-path-1.png}
\scriptsize
\bi
\li positive and negative correlations obvious
\ei
\end{frame}

\bfr{Need to show direction along path}
\fig{1}{house-price-path-bad-1.png}
\end{frame}

\bfr{Connected scatter plot or two line graphs?}
\bi
\li Separate line graphs tend to be easier to read.
\li Once people are used to connected scatter plots they may be able to extract certain patterns (such as cyclical behavior) that can be difficult to spot in line graphs.
\li  Readers are more likely to confuse order and direction in a connected scatter plot than in line graphs and less likely to report correlation.
\li Connected scatter plots seem to result in higher engagement, and thus such plots may be a effective tools to draw readers into a story.
\ei
\end{frame}

\bfr{}
\cola{0.45}{\bf CSP with PCA}


 Recession vs. recovery


\colb{0.65}
\fig{1}{fred-md-PCA-1.png}
\colc
\end{frame}

\end{document}

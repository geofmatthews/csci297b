\documentclass{beamer}
\usetheme{Singapore}
\usepackage{changepage}

%\usepackage{pstricks,pst-node,pst-tree}
\usepackage{amssymb,latexsym}
\usepackage{tikz}
\usepackage{graphicx}
\usepackage{fancyvrb}
\usepackage{hyperref}
\usepackage{fancybox}
\usepackage[listings]{tcolorbox}

\definecolor{codegreen}{rgb}{0,0.6,0}
\definecolor{codegray}{rgb}{0.5,0.5,0.5}
\definecolor{codepurple}{rgb}{0.58,0,0.82}
\definecolor{backcolour}{rgb}{0.95,0.95,0.92}

\lstdefinestyle{mystyle}{
    language=Python,
    backgroundcolor=\color{backcolour},   
    commentstyle=\color{codegreen},
    keywordstyle=\color{magenta},
    numberstyle=\tiny\color{codegray},
    stringstyle=\color{codepurple},
    basicstyle=\ttfamily\scriptsize,
    breakatwhitespace=false,         
    breaklines=true,                 
    captionpos=b,                    
    keepspaces=true,                 
    numbers=left,                    
    numbersep=5pt,                  
    showspaces=false,                
    showstringspaces=false,
    showtabs=false,                  
    tabsize=2,
    escapechar=|,
    frame=single
}

\lstset{style=mystyle}


\newcommand{\bi}{\begin{itemize}}
\newcommand{\li}{\item}
\newcommand{\ei}{\end{itemize}}
\newcommand{\Show}[1]{
\begin{center}
\shadowbox{\begin{minipage}{0.8\textwidth}
          #1
          \end{minipage}}
\end{center}
}
\newcommand{\arrow}{\ensuremath{\rightarrow}}

\newcommand{\uparr}{\ensuremath{\uparrow}}


\newcommand{\fig}[2]{\centerline{\includegraphics[width=#1\textwidth]{#2}}}

\newcommand{\bfr}[1]{\begin{frame}[fragile]\frametitle{{ #1 }}}
\newcommand{\efr}{\end{frame}}

\newcommand{\cola}[1]{\begin{columns}\begin{column}{#1\textwidth}}
\newcommand{\colb}[1]{\end{column}\begin{column}{#1\textwidth}}
\newcommand{\colc}{\end{column}\end{columns}}


\title{Fundamentals of Data Visualization}
\author{Chapter 5}

\begin{document}

\begin{frame}
\maketitle

\end{frame}

\bfr{Taxonomy of visualizations}

\bi
\li Amounts
\li Distributions
\li Proportionas

\li $x$-$y$ relationships
\li Geospatial data
\li Uncertainty
\ei

\end{frame}

\bfr{Visualizing amounts}

\bi
\li Most popular are bars or dots
\ei

\fig{1}{amounts-1.png}

\end{frame}

\bfr{Bars for more sets of categories}
\bi
\li Grouped bars
\li Stacked bars
\li Heat map
\ei

\fig{1}{amounts_multi-1.png}

\end{frame}

\bfr{Distributions}
\fig{1}{single-distributions-1.png}
\bi
\li Histograms and density plots are intuitive visualizations
\li Both require arbitrary parameter choices and can be misleading
\li Cumulative density and Q-Q plots always represent the data faithfully
\li CD and Q-Q can be difficult to interpret
\ei
\end{frame}

\bfr{Multiple distributions}
\fig{.8}{multiple-distributions-1.png}
\small
\bi
\li Boxplots, violins, strip charts, and sina plots are useful for many distributions
at once.
\li Good at showing overall shifts among distributions.
\li Stacked histograms and overlapping densities allow a more in-depth comparison of a smaller number of distributions.
\li Stacked histograms are difficult to interpret and best avoided.
\ei
\end{frame}

\bfr{Proportions}
\fig{1}{proportions-1.png}
\bi
\li Pie charts emphasize that the parts add up to a whole.
\li Pie charts are difficult to compare individual sizes.
\li Stacked bars look awkward for a single set of proportions
\li Can be useful when comparing multiple sets of proportions
\ei
\end{frame}

\bfr{Multiple Proportions}
\fig{1}{proportions-comp-1.png}
\bi
\li Pie charts tend to be space-inefficient and often obscure relationships.
\li Grouped bars work well as long as the number of conditions compared is moderate
\li Stacked bars can work for large numbers of conditions
\li Stacked densities are appropriate when the proportions change along a continuous variable.
\ei
\end{frame}

\bfr{Proportions with multiple grouping variables}

\fig{1}{proportions-multi-1.png}
\bi
\li Mosaic plots, treemaps, and parallel sets can visualize
proportions according to multiple grouping variables.
\li Mosaic plots assume every level of one grouping variable
can be combined with every level of anothre grouping variable.
\li Treemaps do not make that assumption.
\li Parallel set work better when there are more than
two grouping variables.
\ei
\end{frame}

\bfr{$x$-$y$ relationships}

\fig{1}{basic-scatter-1.png}
\bi
\li Scatterplots are the archetypical visualization when we want to show one quantitative variable relative to another
\li If we have three quantitative variables, we can map one onto the dot size, creating  the  bubble chart
\li For paired data, where the variables along the $x$ and the $y$ axes are measured in the same units, it is generally helpful to add a line indicating $x=y$
\li Paired data can also be shown as a slope graph of paired points connected by straight lines
\ei
\end{frame}

\bfr{Large numbers of points}

\fig{1}{xy-binning-1.png}
\bi
\li For large numbers of points, regular scatterplots can become uninformative due to overplotting.
\li Contour lines, 2D bins, or hex bins may provide an alternative
\li For more than two quantities, we may want to plot correlation
coefficients in a correlogram
\ei

\end{frame}

\bfr{Lines}
\fig{1}{xy-lines-1.png}
\bi
\li When the $x$ axis represents time or a strictly increasing quantity such as a treatment dose, we commonly draw line graphs 
\li If we have a temporal sequence of two response variables, we can draw a connected scatterplot
\li We can use smooth lines to represent trends in a larger dataset
\ei
\end{frame}

\bfr{Geospatial data}

\fig{.9}{geospatial-1.png}
\bi
\li The primary mode of showing geospatial data is in the form of a map
\li  A map takes coordinates on the globe and projects them onto a flat surface
\li Data values in different regions are shown by different colors.
\li Such a map is called a choropleth 
\li In some cases, it may be helpful to distort the different regions according to some other quantity (e.g., population number) or simplify each region into a square. 
\li Such visualizations are called cartograms.
\ei
\end{frame}

\bfr{Uncertainty}

\fig{1}{errorbars-1.png}

\bi
\li Error bars are meant to indicate the range of likely values for some estimate or measurement
\li E.g. mean and standard deviation, or median and inter-quartile range
\li Graded error bars show multiple ranges at the same time, where each range corresponds to a different degree of confidence. 
\li They are in effect multiple error bars with different line thicknesses plotted on top of each other.
\ei

\end{frame}

\bfr{More detail than error bars}
\fig{1}{confidence-dists-1.png}

\bi
\li  We can visualize the actual confidence or posterior distributions 
\li  Confidence strips provide a clear visual sense of uncertainty but are difficult to read accurately.
\li  Eyes and half-eyes combine error bars with approaches to visualize distributions (violins and ridgelines, respectively)
\li By showing the distribution in discrete units, the quantile dot plot is not as precise but can be easier to read than the continuous distribution shown by a violin or ridgeline plot.
\ei

\end{frame}

\bfr{Confidence bands}
\fig{1}{confidence-bands-1.png}

\bi
\li For smooth line graphs, the equivalent of an error bar is a confidence band

\ei


\end{frame}
\end{document}

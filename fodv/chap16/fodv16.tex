\documentclass{beamer}
\usetheme{Singapore}
\usepackage{changepage}

%\usepackage{pstricks,pst-node,pst-tree}
\usepackage{amssymb,latexsym}
\usepackage{tikz}
\usepackage{graphicx}
\usepackage{fancyvrb}
\usepackage{hyperref}
\usepackage{fancybox}

\newcommand{\bi}{\begin{itemize}}
\newcommand{\li}{\item}
\newcommand{\ei}{\end{itemize}}
\newcommand{\Show}[1]{
\begin{center}
\shadowbox{\begin{minipage}{0.8\textwidth}
          #1
          \end{minipage}}
\end{center}
}
\newcommand{\arrow}{\ensuremath{\rightarrow}}

\newcommand{\uparr}{\ensuremath{\uparrow}}


\newcommand{\fig}[2]{\centerline{\includegraphics[width=#1\textwidth]{#2}}}

\newcommand{\figg}[2]{\includegraphics[width=#1\textwidth]{#2}}

\newcommand{\bfr}[1]{\begin{frame}[fragile]\frametitle{{ #1 }}}
\newcommand{\efr}{\end{frame}}

\newcommand{\cola}[1]{\begin{columns}\begin{column}{#1\textwidth}}
\newcommand{\colb}[1]{\end{column}\begin{column}{#1\textwidth}}
\newcommand{\colc}{\end{column}\end{columns}}


\author{Fundamentals of Data Visualization}
\title{Chapter 16\\Visualizing uncertainty}

\begin{document}

\begin{frame}
\maketitle
\end{frame}

\bfr{Visualizing uncertainty}
\bi
\li Error bars and confidence bands are traditional.
\li Difficult to interpret correctly.
\li Precise and space efficient.
\li For lay audience, other approaches may be preferable.
\ei
\end{frame}

\bfr{Probability interpretations}
\bi
\li Physical, objective, or frequentist
\bi
\li associated with random variables: roulette wheels, coin flips, physical measurements.
\li relative frequency
\li propensity
\ei
\li Evidential, or Bayesian
\bi
\li associated with any statement whatever
\li degree to which it is supported by the evidence
\li disposition to gamble at certain odds
\ei
\ei
\end{frame}

\bfr{Hard to visualize}
\fig{.7}{dieprobability}
\bi
\li Plotting probabilities as numbers.
\li General public has difficulty interpreting this number.
\ei
\end{frame}

\bfr{Discrete outcome visualization}
\fig{1}{probability-waffle-1.png}
\bi
\li Frequency framing
\ei
\end{frame}

\bfr{Election predictions}
\fig{1}{election-prediction-1.png}
\scriptsize
\bi
\li Blue party is predicted to have a one percentage point advantage over the yellow party, with a margin of error (95\%) of 1.76 percentage points.

\li What are the probabilities of blue and yellow winning? 
\li Probability of blue winning is area under the curve, 87.1\%
\ei
\end{frame}

\bfr{Quantile dotplot}
\fig{.8}{election-quantile-dot-1.png}
\bi
\li Humans are much better at counting than judging area.
\li Don't use too many dots.
\ei
\end{frame}


\bfr{Statistical sampling}
\fig{.9}{sampling-schematic-1.png}
\end{frame}

\bfr{Five different error bars}
\fig{1}{cocoa-data-vs-CI-1.png}
\bi
\li \bf Whenever you visualize uncertainty with error bars, you must specify what quantity and/or confidence level the error bars represent.
\ei
\end{frame}

\bfr{Sample size determines standard error}
\fig{1}{cocoa-CI-vs-n-1.png}
\bi
\li Graded error bars
\li Shading important to indicate this is a probability
\li Single error bar easily misinterpreted as min and max possible.
\li This is a {\em deterministic construal error.}
\ei
\end{frame}

\bfr{Error bars in scientific publications}
\fig{1}{mean-chocolate-ratings-1.png}
\bi
\li Difficult to judge {\em significant differences}.
\ei
\end{frame}
\bfr{Confidence intervals for the difference in means}
\fig{1}{chocolate-ratings-contrasts-1.png}
\scriptsize
\bi
\li Only Canada is significantly different from the US.
\li Switzerland is  significantly different at the 95\% level, but not at the 99\% level from the US.
\li There is no evidence at all that Austria or Belgium is significantly different from the US.
\ei
\end{frame}

\bfr{Types of error bars}
\fig{1}{confidence-visualizations-1.png}
\end{frame}

\bfr{Error bars on bar plots}
\fig{1}{butterfat-bars-1.png}
\end{frame}

\bfr{Two dimensional error bars}
\fig{1}{median-age-income-1.png}
\end{frame}

\bfr{Bayesians vs. Frequentists}
\bi
\li Frequentists asses uncertainty with {\bf confidence intervals.}
\li Bayesians calculate {\bf posterior distributions} and {\bf credible intervals}.
\li The credible interval indicates a range of values in which the parameter value is expected with a given probability, as calculated from the posterior distribution.
\li For example, a 95\% credible interval corresponds to the center 95\% of the posterior distribution.
\li The true parameter value has a 95\% chance of lying in the 95\% credible interval.
\ei
\end{frame}

\bfr{Bayesians vs. Frequentists}
\bi
\li Bayesians calculate where a parameter is.
\li Frequentists calculate where a parameter is not.
\ei
\end{frame}

\bfr{Bayesians vs. Frequentists}
\bi
\li Under the Bayesian approach, you use the data and your prior knowledge about the system under study (called the prior) to calculate a probability distribution (the posterior) that tells you where you can expect the true parameter value to lie.
\ei
\end{frame}

\bfr{Bayesians vs. Frequentists}
\bi
\li By contrast, under the frequentist approach, you first make an assumption that you intend to disprove. 
\li This assumption is called the null hypothesis, and it is often simply the assumption that the parameter equals zero (e.g., there is no difference between two conditions). 
\li You then calculate the probability that random sampling would generate data similar to what was observed if the null hypothesis were true.
\li The confidence interval is a representation of this probability.
\li If a given confidence interval excludes the parameter value under the null hypothesis (i.e., the value zero), then you can reject the null hypothesis at that confidence level.
\ei
\end{frame}

\begin{frame}
\fig{.8}{ci-frequentist-expl-1.png}
\end{frame}


\bfr{Bayesians vs. Frequentists}
\bi
\li A Bayesian credible interval makes a statement about the true parameter value and a frequentist confidence interval makes a statement about the null hypothesis.
\li In practice, there is little difference.
\li The Bayesian approach  emphasizes thinking about the magnitude of an effect
\li The frequentist approach emphasizes a binary perspective of an effect either existing or not.

\ei
\end{frame}

\bfr{Bayesians vs. Frequentists in practice}
\fig{1}{bayes-vs-ols-1.png}
\end{frame}


\bfr{Bayesians vs. Frequentists}

\bi
\li A {\bf Bayesian credible interval} answers the question:\\ “Where do we expect the true parameter value to lie?” 
\bigskip
\li A {\bf frequentist confidence interval} answers the question:\\ “How certain are we that the true parameter value is not zero?”
\ei
\end{frame}

\bfr{Bayesians posterior distributions}\scriptsize
\fig{1}{bayes-ridgeline-1.png}
\bi
\li Bayesians infer an entire distribution, not just a confidence interval.
\li All the approaches to visualizing distributions are available.
\ei
\end{frame}

\bfr{Curve fit uncertainty}\scriptsize
\fig{.8}{blue-jays-male-conf-band-1.png}
\bi
\li 95\% confident the true line lies in the gray area
\ei
\end{frame}

\bfr{Random lines from the posterior distribution}
\fig{.8}{blue-jays-male-fitted-draws-1.png}
\bi
\li Uncertainty comes from slope and intercept
\ei
\end{frame}

\bfr{Multiple confidence intervals}
\fig{.8}{blue-jays-male-graded-conf-band-1.png}
\end{frame}

\bfr{Non-linear model fits}
\fig{1}{mpg-uncertain-1.png}
\bi
\li Uncertainty comes from wiggliness as well.
\ei
\end{frame}

\bfr{Animating uncertainty}
\bi
\li
\url{https://clauswilke.com/dataviz/visualizing-uncertainty.html}
\ei
\end{frame}


\end{document}

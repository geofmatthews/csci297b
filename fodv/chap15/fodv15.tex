\documentclass{beamer}
\usetheme{Singapore}
\usepackage{changepage}

%\usepackage{pstricks,pst-node,pst-tree}
\usepackage{amssymb,latexsym}
\usepackage{tikz}
\usepackage{graphicx}
\usepackage{fancyvrb}
\usepackage{hyperref}
\usepackage{fancybox}

\newcommand{\bi}{\begin{itemize}}
\newcommand{\li}{\item}
\newcommand{\ei}{\end{itemize}}
\newcommand{\Show}[1]{
\begin{center}
\shadowbox{\begin{minipage}{0.8\textwidth}
          #1
          \end{minipage}}
\end{center}
}
\newcommand{\arrow}{\ensuremath{\rightarrow}}

\newcommand{\uparr}{\ensuremath{\uparrow}}


\newcommand{\fig}[2]{\centerline{\includegraphics[width=#1\textwidth]{#2}}}

\newcommand{\figg}[2]{\includegraphics[width=#1\textwidth]{#2}}

\newcommand{\bfr}[1]{\begin{frame}[fragile]\frametitle{{ #1 }}}
\newcommand{\efr}{\end{frame}}

\newcommand{\cola}[1]{\begin{columns}\begin{column}{#1\textwidth}}
\newcommand{\colb}[1]{\end{column}\begin{column}{#1\textwidth}}
\newcommand{\colc}{\end{column}\end{columns}}


\author{Fundamentals of Data Visualization}
\title{Chapter 14}

\begin{document}

\begin{frame}
\maketitle
\end{frame}

\bfr{Visualizing geospatial data}
\bi
\li Many datasets contain information linked to locations:
\bi
\li where specific plants or animals have been found 
\li  where people with specific attributes (such as income, age, or educational attainment) live
\li where man-made objects (e.g., bridges, roads, buildings) have been constructed 
\ei
\li
Maps tend to be intuitive to readers but they can be challenging to design. 
\li The choropleth map represents data values as differently colored spatial areas.
\li  Cartograms purposefully distort map areas or represent them in stylized form, for example as equal-sized squares.
\ei
\end{frame}

\bfr{World coordinates}
\cola{0.4}
\fig{1.5}{world-orthographic-1.png}
\colb{0.6}
\bi
\li To specify a place:\\ latitude, longitude, altitude
\li A reference system for these is called a {\bf datum}
\li One widely used datum is the World Geodetic System (WGS) 84, which is used by the Global Positioning System (GPS).
\ei
\colc
\end{frame}

\bfr{World coordinates}
\cola{0.4}
\fig{1.5}{world-orthographic-1.png}
\colb{0.6}
\bi
\li Frequently altitude is not recorded.
\li Lines of equal longitude are {\bf meridians}
\li Lines of equal latitude are called {\bf parallels}
\li The {\bf prime meridian} is at $0^\circ$ longitude
\li The {\bf equator} is at $0^\circ$ latitude
\ei
\colc
\end{frame}

\bfr{Projections}
\bi
\li In map-making we need to take the spherical surface of the earth and flatten it out.
\li This  projection  introduces distortions.
\li The projection can preserve either angles or areas. 
\li A projection that preserves angles is called {\bf conformal}.
\li A projection that preserves areas is called {\bf equal-area}.
\li  Other projections may  instead preserve other quantities of interest, such as distances to some reference point or line. 
\li Some projections attempt to strike a compromise between preserving angles and areas.
\li These compromise projections are frequently used to display the entire world in an aesthetically pleasing manner.
\ei
\end{frame}

\bfr{Mercator: a conformal projection}
\fig{1}{world-mercator-1.png}
\end{frame}

\bfr{Goode homolosine: an equal-area projection}
\fig{1.2}{world-goode-1.png}
\end{frame}

\bfr{A challenge: the USA}
\fig{1}{usa-orthographic-1.png}
\end{frame}

\bfr{Equal-area Albers projection}
\fig{1}{usa-true-albers-1.png}
\end{frame}

\bfr{Common practice}
\fig{1}{usa-albers-1.png}
\end{frame}

\bfr{Preserving areas}
\fig{1}{usa-albers-revised-1.png}
\end{frame}


\end{document}

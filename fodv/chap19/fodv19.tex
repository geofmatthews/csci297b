\documentclass{beamer}
\usetheme{Singapore}
\usepackage{changepage}

%\usepackage{pstricks,pst-node,pst-tree}
\usepackage{amssymb,latexsym}
\usepackage{tikz}
\usepackage{graphicx}
\usepackage{fancyvrb}
\usepackage{hyperref}
\usepackage{fancybox}

\newcommand{\bi}{\begin{itemize}}
\newcommand{\li}{\item}
\newcommand{\ei}{\end{itemize}}
\newcommand{\Show}[1]{
\begin{center}
\shadowbox{\begin{minipage}{0.8\textwidth}
          \bf #1
          \end{minipage}}
\end{center}
}
\newcommand{\arrow}{\ensuremath{\rightarrow}}

\newcommand{\uparr}{\ensuremath{\uparrow}}


\newcommand{\fig}[2]{\centerline{\includegraphics[width=#1\textwidth]{#2}}}

\newcommand{\figg}[2]{\includegraphics[width=#1\textwidth]{#2}}

\newcommand{\bfr}[1]{\begin{frame}[fragile]\frametitle{{ #1 }}}
\newcommand{\efr}{\end{frame}}

\newcommand{\cola}[1]{\begin{columns}\begin{column}{#1\textwidth}}
\newcommand{\colb}[1]{\end{column}\begin{column}{#1\textwidth}}
\newcommand{\colc}{\end{column}\end{columns}}


\author{Fundamentals of Data Visualization}
\title{Chapter 19}

\begin{document}

\begin{frame}
\maketitle
\end{frame}

\begin{frame}
\cola{0.6}
\fig{1}{popgrowth-vs-popsize-colored-1.png}
\colb{0.4}
\bi
\li Pitfalls of color use:  \li too much information.
\ei
\colc
\end{frame}

\bfr{Direct labelling}
\fig{1}{popgrowth-vs-popsize-bw-1.png}
\bi
\li Use direct labeling instead of colors when you need to distinguish between more than about eight categorical items.
\ei
\end{frame}

\begin{frame}
\cola{0.6}
\fig{1}{popgrowth-US-rainbow-1.png}
\colb{0.4}
\bi
\li Pitfalls of color use:  \bi\li color with no purpose.\ei
\li Avoid large filled areas of overly saturated colors. They make it difficult for your reader to carefully inspect your figure.
\ei
\colc
\end{frame}

\bfr{Non-monotonic color scales}
\fig{1}{rainbow-desaturated-1.png}
\bi
\li Colors at two ends are the same.
\li Areas where colors change rapidly.
\li Areas where colors change slowly.
\li Grayscale shows the brightness changes randomly.
\ei
\end{frame}

\bfr{Rainbow scale}
\fig{.8}{map-Texas-rainbow-1.png}
\bi
\li Painful to look at for a long time.
\li Highlights some things, obscures others.
\ei
\end{frame}

\bfr{Color blindness}
\bi
\li  deuteranomaly/deuteranopia: difficulty seeing green 
\li  protanomaly/protanopia: difficulty seeing red
\li tritanomaly/tritanopia: difficulty seeing blue 
\li  anomaly:  some impairment in the perception 
\li anopia: complete absence of perception of that color
\li Approximately 8\% of males and 0.\% of females suffer from some sort of color-vision deficiency
\li red/green deuteranomoly is most common form
\ei
\end{frame}


\end{document}

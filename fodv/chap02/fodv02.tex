\documentclass{beamer}
\usetheme{Singapore}
\usepackage{changepage}

%\usepackage{pstricks,pst-node,pst-tree}
\usepackage{amssymb,latexsym}
\usepackage{tikz}
\usepackage{graphicx}
\usepackage{fancyvrb}
\usepackage{hyperref}
\usepackage{fancybox}
\usepackage[listings]{tcolorbox}

\definecolor{codegreen}{rgb}{0,0.6,0}
\definecolor{codegray}{rgb}{0.5,0.5,0.5}
\definecolor{codepurple}{rgb}{0.58,0,0.82}
\definecolor{backcolour}{rgb}{0.95,0.95,0.92}

\lstdefinestyle{mystyle}{
    language=Python,
    backgroundcolor=\color{backcolour},   
    commentstyle=\color{codegreen},
    keywordstyle=\color{magenta},
    numberstyle=\tiny\color{codegray},
    stringstyle=\color{codepurple},
    basicstyle=\ttfamily\scriptsize,
    breakatwhitespace=false,         
    breaklines=true,                 
    captionpos=b,                    
    keepspaces=true,                 
    numbers=left,                    
    numbersep=5pt,                  
    showspaces=false,                
    showstringspaces=false,
    showtabs=false,                  
    tabsize=2,
    escapechar=|,
    frame=single
}

\lstset{style=mystyle}


\newcommand{\bi}{\begin{itemize}}
\newcommand{\li}{\item}
\newcommand{\ei}{\end{itemize}}
\newcommand{\Show}[1]{
\begin{center}
\shadowbox{\begin{minipage}{0.8\textwidth}
          #1
          \end{minipage}}
\end{center}
}
\newcommand{\arrow}{\ensuremath{\rightarrow}}

\newcommand{\uparr}{\ensuremath{\uparrow}}


\newcommand{\fig}[2]{\centerline{\includegraphics[width=#1\textwidth]{#2}}}

\newcommand{\bfr}[1]{\begin{frame}[fragile]\frametitle{{ #1 }}}
\newcommand{\efr}{\end{frame}}

\newcommand{\cola}[1]{\begin{columns}\begin{column}{#1\textwidth}}
\newcommand{\colb}[1]{\end{column}\begin{column}{#1\textwidth}}
\newcommand{\colc}{\end{column}\end{columns}}


\title{Fundamentals of Data Visualization}
\author{Chapter 2}

\begin{document}

\begin{frame}
\maketitle
\end{frame}

\bfr{Mapping data onto aesthetics}

\bi
\li
On first glance a scatter plot, a pie chart, and a heatmap don’t seem to have much in common.
\li
All data visualizations map data values into quantifiable features of the resulting graphic. 
\li
We refer to these features as aesthetics.
\ei

\end{frame}

\bfr{Common aesthetics}
\fig{1.0}{common-aesthetics-1}
\end{frame}

\bfr{Other aesthetics}
\bi
\li Font family, size.
\li Transparency of overlapping aesthetics.
\ei
\end{frame}

\bfr{Two fundamental types}
\bi
\li Continuous data
\bi
\li e.g. time duration
\li position, size, color, line width
\ei
\li Everything else
\bi
\li e.g. gender
\li shape, line type
\ei
\ei
\end{frame}

\bfr{Types of data}
\scriptsize
\begin{tabular}{lll}
Type of variable&	Examples&	Appropriate scale	\\\hline
quantitative/numerical continuous	&1.3, 5.7, 83, 1.5x10-2	& continuous	\\
quantitative/numerical discrete&	1, 2, 3, 4	&discrete	\\
qualitative/categorical unordered&	dog, cat, fish&	discrete\\
qualitative/categorical ordered &	good, fair, poor	&discrete	 \\
date or time	&Jan. 5 2018, 8:03am&	continuous or discrete	\\
text &	``No comment.''&	none, or discrete	\\
\end{tabular}


\end{frame}

\bfr{Example Data Frame}
\begin{tabular}{lllcc}
Month	 &Day	 &Location	 &Station ID	 &Temperature\\\hline
Jan	 &1	 &Chicago	 &USW00014819	 &25.6\\
Jan	 &1	 &San Diego	 &USW00093107	 &55.2\\
Jan	 &1	 &Houston	 &USW00012918	 &53.9\\
Jan	 &1	 &Death Valley	 &USC00042319	 &51.0\\
Jan	 &2	 &Chicago	 &USW00014819	 &25.5\\
Jan	 &2	 &San Diego	 &USW00093107	 &55.3\\
Jan	 &2	 &Houston	 &USW00012918	 &53.8\\
Jan	 &2	 &Death Valley	 &USC00042319	 &51.2\\
Jan	 &3	 &Chicago	 &USW00014819	 &25.3
\end{tabular}
\end{frame}

\bfr{Scales map data values onto aesthetics}

\fig{1}{basic-scales-example-1}
\end{frame}

\bfr{Daily temperature normals}
\fig{1}{temp-normals-vs-time-1}
\end{frame}

\bfr{Daily temperature normals}
\fig{1}{four-locations-temps-by-month-1}

\bi
\li Vertical scale is unordered discrete
\bi\li order chosen by eye \ei
\li Horizontal scale is ordered discrete
\ei
\end{frame}

\bfr{Five scales on one graph}
\fig{0.8}{mtcars-five-scale-1}
\end{frame}
\end{document}

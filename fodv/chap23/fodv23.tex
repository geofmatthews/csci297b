\documentclass{beamer}
\usetheme{Singapore}
\usepackage{changepage}

%\usepackage{pstricks,pst-node,pst-tree}
\usepackage{amssymb,latexsym}
\usepackage{tikz}
\usepackage{graphicx}
\usepackage{fancyvrb}
\usepackage{hyperref}
\usepackage{fancybox}

\newcommand{\bi}{\begin{itemize}}
\newcommand{\li}{\item}
\newcommand{\ei}{\end{itemize}}
\newcommand{\Show}[1]{
\begin{center}
\shadowbox{\begin{minipage}{0.8\textwidth}
          \bf #1
          \end{minipage}}
\end{center}
}
\newcommand{\arrow}{\ensuremath{\rightarrow}}

\newcommand{\uparr}{\ensuremath{\uparrow}}


\newcommand{\fig}[2]{\centerline{\includegraphics[width=#1\textwidth]{#2}}}

\newcommand{\figg}[2]{\includegraphics[width=#1\textwidth]{#2}}

\newcommand{\bfr}[1]{\begin{frame}[fragile]\frametitle{{ #1 }}}
\newcommand{\efr}{\end{frame}}

\newcommand{\cola}[1]{\begin{columns}\begin{column}{#1\textwidth}}
\newcommand{\colb}[1]{\end{column}\begin{column}{#1\textwidth}}
\newcommand{\colc}{\end{column}\end{columns}}


\author{Fundamentals of Data Visualization}
\title{Chapter 23\\ Balance the data and the context}

\begin{document}

\begin{frame}
\maketitle
\end{frame}

\bfr{Two parts of a graph}
\bi
\li Parts that represent data
\bi \li bars \li dots \li lines \li shaded areas \ei
\li Parts that don't
\bi \li axes \li titles \li labels\li legends \li annotations \ei
\ei

\bigskip
\Show{Maximize the data–ink ratio, within reason.}
\end{frame}

\bfr{Too much non-data ink}
\fig{1}{Aus-athletes-grid-bad-1.png}
\end{frame}

\bfr{Too little non-data ink}
\fig{1}{Aus-athletes-min-bad-1.png}
\end{frame}

\bfr{Goldilocks non-data ink}
\fig{1}{Aus-athletes-grid-good-1.png}
\end{frame}


\bfr{Frame is optional}
\fig{1}{Aus-athletes-grid-good-frame-1.png}
\end{frame}

\bfr{Background grids}
\cola{0.5}
\fig{1}{titanic-survival-by-gender-class-bad-1.png}
\colb{0.5}
\fig{1}{titanic-survival-by-gender-class-1.png}
\colc
\end{frame}

\bfr{Background grids}
\fig{0.8}{price-plot-ggplot-default-1.png}\scriptsize
\bi
\li a little busy and distracting
\li  helps the plot to be perceived as a single visual entity 
\li prevents the plot to appear as a white box in surrounding dark text 
\li really depends on the formatting of the text and the design of the figure
\ei
\end{frame}

\bfr{Background grids}
\fig{0.8}{price-plot-no-grid-1.png}
\bi
\li   data lines not sufficiently anchored
\ei
\end{frame}



\bfr{Data normalized to $y=100$}
\fig{0.8}{price-plot-refline-1.png}

\end{frame}


\bfr{Better set of reference lines}
\fig{0.8}{price-plot-hgrid-1.png}

\end{frame}

\bfr{Reference lines perpendicular to data}
\fig{0.8}{price-increase-1.png}
\bi\li Grid lines that run perpendicular to the key variable of interest tend to be the most useful.\ei
\end{frame}

\bfr{Grid lines on the bars}
\fig{0.8}{price-increase-tufte-1.png}
\bi
\li helps the reader to perceive bar lengths
\li the bars look like they are falling apart and don’t form a clear visual unit
\li  have to move the percentage values outside the bars
\ei
\end{frame}

\bfr{Paired data}
\cola{.5}
\fig{1}{gene-expression-1.png}
\colb{.5}
\fig{1}{gene-expression-bad-1.png}
\colc
\bigskip

\centerline{ $45^\circ$ line better than grid}
\end{frame}

\bfr{Paired data}
\fig{.7}{gene-expression-ugly-1.png}

\centerline{Too busy?}
\end{frame}

\bfr{The right amount of context}

\bi
\li Both overloading a figure with non-data ink and excessively erasing non-data ink can result in poor figure design.
\li We need to find a healthy medium:
\bi
\li the data points are the main emphasis of the figure
\li   sufficient context is provided about what data is shown
\ei
\ei
\end{frame}

\bfr{Background grids}
\bi
\li Think carefully about which specific grid or guide lines are most informative for the plot you are making, and then only show those. 
\li Minimal, light grids on a white background recommended
\li A shaded background can help the plot appear as a single visual entity
\ei
\end{frame}


\end{document}

\documentclass{beamer}
\usetheme{Singapore}
\usepackage{changepage}

%\usepackage{pstricks,pst-node,pst-tree}
\usepackage{amssymb,latexsym}
\usepackage{tikz}
\usepackage{graphicx}
\usepackage{fancyvrb}
\usepackage{hyperref}
\usepackage{fancybox}
\usepackage[listings]{tcolorbox}

\definecolor{codegreen}{rgb}{0,0.6,0}
\definecolor{codegray}{rgb}{0.5,0.5,0.5}
\definecolor{codepurple}{rgb}{0.58,0,0.82}
\definecolor{backcolour}{rgb}{0.95,0.95,0.92}

\lstdefinestyle{mystyle}{
    language=Python,
    backgroundcolor=\color{backcolour},   
    commentstyle=\color{codegreen},
    keywordstyle=\color{magenta},
    numberstyle=\tiny\color{codegray},
    stringstyle=\color{codepurple},
    basicstyle=\ttfamily\scriptsize,
    breakatwhitespace=false,         
    breaklines=true,                 
    captionpos=b,                    
    keepspaces=true,                 
    numbers=left,                    
    numbersep=5pt,                  
    showspaces=false,                
    showstringspaces=false,
    showtabs=false,                  
    tabsize=2,
    escapechar=|,
    frame=single
}

\lstset{style=mystyle}


\newcommand{\bi}{\begin{itemize}}
\newcommand{\li}{\item}
\newcommand{\ei}{\end{itemize}}
\newcommand{\Show}[1]{
\begin{center}
\shadowbox{\begin{minipage}{0.8\textwidth}
          #1
          \end{minipage}}
\end{center}
}
\newcommand{\arrow}{\ensuremath{\rightarrow}}

\newcommand{\uparr}{\ensuremath{\uparrow}}


\newcommand{\fig}[2]{\centerline{\includegraphics[width=#1\textwidth]{#2}}}

\newcommand{\bfr}[1]{\begin{frame}[fragile]\frametitle{{ #1 }}}
\newcommand{\efr}{\end{frame}}

\newcommand{\cola}[1]{\begin{columns}\begin{column}{#1\textwidth}}
\newcommand{\colb}[1]{\end{column}\begin{column}{#1\textwidth}}
\newcommand{\colc}{\end{column}\end{columns}}


\title{Fundamentals of Data Visualization}
\author{Chapter 4}

\begin{document}

\begin{frame}
\maketitle
\end{frame}

\bfr{Color as a tool to distinguish}
\bi
\li Discrete items that do not have an intrinsic order.
\bi\li coutries on a map\li manufacturere of a product\ei
\li A finite set of specific colors chosen to:
\li \ \ look clearly distinct from each other, but
\li \ \ not stand out from each other, and
\li \ \ not create the impression of an order.
\ei

\fig{1}{qualitative-scales-1.png}

\end{frame}

\begin{frame}
\fig{.7}{popgrowth-US-1.png}
\end{frame}


\bfr{Color to represent data values}
\bi
\li Sequential color scale
\li Which values are larger or smaller
\li How distant the values are from each other
\bi\li scale needs to be perceived to vary uniformly\ei
\li Can be based on single hue or multiple hues
\ei

\fig{1}{sequential-scales-1.png}

\end{frame}

\begin{frame}
\fig{1}{map-Texas-income-1.png}
\end{frame}


\bfr{Color to represent deviation from norm}
\bi
\li Diverging color scale
\li Neutral is usually a light color
\li Two extremes are contrasting colors
\li Extremes should be balanced in value
\ei

\fig{1}{diverging-scales-1.png}

\end{frame}

\begin{frame}
\fig{1}{map-Texas-race-1.png}
\end{frame}

\bfr{Color as a tool to highlight}
\bi
\li Some colors vividly stand out 
\li Accent color scales
\bi
\li a set of subdued colors
\li a matching set of stronger, darker, more saturated colors
\ei
\ei

\fig{1}{accent-scales-1.png}

\end{frame}

\begin{frame}
\fig{.7}{popgrowth-US-highlight-1.png}
\end{frame}


\bfr{Baseline colors should not compete for attention}
\bi
\li Pick drab colors
\li Pick no color
\ei

\fig{1}{Aus-athletes-track-1.png}

\end{frame}

\end{document}

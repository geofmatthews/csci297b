\documentclass{beamer}
\usetheme{Singapore}
\usepackage{changepage}

%\usepackage{pstricks,pst-node,pst-tree}
\usepackage{amssymb,latexsym}
\usepackage{tikz}
\usepackage{graphicx}
\usepackage{fancyvrb}
\usepackage{hyperref}
\usepackage{fancybox}
\usepackage[listings]{tcolorbox}

\definecolor{codegreen}{rgb}{0,0.6,0}
\definecolor{codegray}{rgb}{0.5,0.5,0.5}
\definecolor{codepurple}{rgb}{0.58,0,0.82}
\definecolor{backcolour}{rgb}{0.95,0.95,0.92}

\lstdefinestyle{mystyle}{
    language=Python,
    backgroundcolor=\color{backcolour},   
    commentstyle=\color{codegreen},
    keywordstyle=\color{magenta},
    numberstyle=\tiny\color{codegray},
    stringstyle=\color{codepurple},
    basicstyle=\ttfamily\scriptsize,
    breakatwhitespace=false,         
    breaklines=true,                 
    captionpos=b,                    
    keepspaces=true,                 
    numbers=left,                    
    numbersep=5pt,                  
    showspaces=false,                
    showstringspaces=false,
    showtabs=false,                  
    tabsize=2,
    escapechar=|,
    frame=single
}

\lstset{style=mystyle}


\newcommand{\bi}{\begin{itemize}}
\newcommand{\li}{\item}
\newcommand{\ei}{\end{itemize}}
\newcommand{\Show}[1]{
\begin{center}
\shadowbox{\begin{minipage}{0.8\textwidth}
          #1
          \end{minipage}}
\end{center}
}
\newcommand{\arrow}{\ensuremath{\rightarrow}}

\newcommand{\uparr}{\ensuremath{\uparrow}}


\newcommand{\fig}[2]{\centerline{\includegraphics[width=#1\textwidth]{#2}}}

\newcommand{\figg}[2]{\includegraphics[width=#1\textwidth]{#2}}

\newcommand{\bfr}[1]{\begin{frame}[fragile]\frametitle{{ #1 }}}
\newcommand{\efr}{\end{frame}}

\newcommand{\cola}[1]{\begin{columns}\begin{column}{#1\textwidth}}
\newcommand{\colb}[1]{\end{column}\begin{column}{#1\textwidth}}
\newcommand{\colc}{\end{column}\end{columns}}


\author{Fundamentals of Data Visualization}
\title{Chapter 10}

\begin{document}

\begin{frame}
\maketitle
\end{frame}

\bfr{Visualizing proportions}
\bi
\li Examples:
\bi
\li the proportions of men and women in a group of people
\li the percentages of people voting for different political parties in an election
\li the market shares of companies.
\ei
\li The archetypal such visualization is the pie chart, omnipresent in any business presentation and much maligned among data scientists.
\li There is no single ideal visualization that always works.
\li \bf You always need to pick the visualization that best fits your specific dataset and that highlights the key data features you want to show.
\ei
\end{frame}

\bfr{German parliament, 1976-1980}
\fig{1}{bundestag-pie-1.png}
\end{frame}



\bfr{German parliament, 1976-1980}
\fig{1}{bundestag-stacked-bars-1.png}
\end{frame}
\bfr{German parliament, 1976-1980}
\fig{.8}{bundestag-bars-side-by-side-1.png}
\end{frame}

\bfr{What's best?}
\bi
\li Many authors categorically reject pie charts.
\li Others defend the use of pie charts in some applications. 
\li Perhaps none of these visualizations is consistently superior to any other?
\li  Depending on the  story you want to tell, you may want to favor one or the other approach.
\li  In the case of the 8th German Bundestag, a pie chart shows clearly that the ruling coalition of SPD and FDP jointly had a small majority over the CDU/CSU.
\li This fact is not visually obvious in any of the other plots.
\ei
\end{frame}

\bfr{Tradeoffs}
\bi
\li Pie charts work well when the goal is to emphasize simple fractions, such as one-half, one-third, or one-quarter.
\li They also work well when we have very small datasets.
\li A single pie chart looks just fine, but a single column of stacked bars looks awkward.
\li Stacked bars can work for side-by-side comparisons of multiple conditions or in a time series.
\li Side-by-side bars work when we want to directly compare the individual fractions to each other.
\ei
\end{frame}

\bfr{Tradeoffs}\scriptsize
\begin{tabular}{p{1.8in}|ccc}
&	Pie chart	&Stacked bars	&Side-by-side bars\\\hline
Clearly visualizes the data as proportions of a whole	&$\checkmark$&$\checkmark$&$\times$\\
Allows easy visual comparison of the relative proportions	&$\times$&$\times$&$\checkmark$\\
Visually emphasizes simple fractions, such as 1/2, 1/3, 1/4	&$\checkmark$&$\times$&$\times$\\
Looks visually appealing even for very small datasets	&$\checkmark$&$\times$&$\checkmark$\\
Works well when the whole is broken into many pieces	&$\times$&$\times$&$\checkmark$\\
Works well for the visualization of many sets of proportions or time series of proportions&$\times$&$\checkmark$&$\times$\\
\end{tabular}
\end{frame}

\bfr{Pie chart failure}

\fig{1}{marketshare-pies-1.png}
\bi
\li Market share of A is growing and E is shrinking.
\li What else?
\ei
\end{frame}
\bfr{Stacked bar failure}

\fig{.8}{marketshare-stacked-1.png}
\bi
\li Market share of A is growing and E is shrinking.
\li What else?
\ei
\end{frame}


\bfr{Side by side bars}

\fig{.8}{marketshare-side-by-side-1.png}
\bi
\li The story is clear.
\ei
\end{frame}

\bfr{Stacked bars work when there are only two groups}
\fig{1}{women-parliament-1.png}
\centerline{Change in the gender composition of the Rwandan parliament}
\end{frame}


\bfr{Stacked densities}
\fig{.8}{health-vs-age-1.png}
\bi
\li Overall health declines with age.
\li Over half remain in good or excellent health until very old age.
\li Obscures the {\bf amount} of people in favor of percentages.
\ei
\end{frame}

\bfr{Side-by-side densities}
\fig{1}{health-vs-age-facets-1.png}
\bi
\li Number in good and excellent declines with age.
\li Number in fair remains constant.
\ei
\end{frame}

\bfr{Marital status does not work so well}
\fig{1}{marital-vs-age-1.png}
\end{frame}

\bfr{Side-by-side reveals much more}
\fig{1}{marital-vs-age-facets-1.png}
\end{frame}

\bfr{Side-by-side proportions answer different questions}
\fig{1}{marital-vs-age-proportions-1.png}
\end{frame}



\end{document}

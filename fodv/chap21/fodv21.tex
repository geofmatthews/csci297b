\documentclass{beamer}
\usetheme{Singapore}
\usepackage{changepage}

%\usepackage{pstricks,pst-node,pst-tree}
\usepackage{amssymb,latexsym}
\usepackage{tikz}
\usepackage{graphicx}
\usepackage{fancyvrb}
\usepackage{hyperref}
\usepackage{fancybox}

\newcommand{\bi}{\begin{itemize}}
\newcommand{\li}{\item}
\newcommand{\ei}{\end{itemize}}
\newcommand{\Show}[1]{
\begin{center}
\shadowbox{\begin{minipage}{0.8\textwidth}
          \bf #1
          \end{minipage}}
\end{center}
}
\newcommand{\arrow}{\ensuremath{\rightarrow}}

\newcommand{\uparr}{\ensuremath{\uparrow}}


\newcommand{\fig}[2]{\centerline{\includegraphics[width=#1\textwidth]{#2}}}

\newcommand{\figg}[2]{\includegraphics[width=#1\textwidth]{#2}}

\newcommand{\bfr}[1]{\begin{frame}[fragile]\frametitle{{ #1 }}}
\newcommand{\efr}{\end{frame}}

\newcommand{\cola}[1]{\begin{columns}\begin{column}{#1\textwidth}}
\newcommand{\colb}[1]{\end{column}\begin{column}{#1\textwidth}}
\newcommand{\colc}{\end{column}\end{columns}}


\author{Fundamentals of Data Visualization}
\title{Chapter 21\\Multi-panel figures}

\begin{document}

\begin{frame}
\maketitle
\end{frame}

\bfr{Multiple views of the data}
\bi
\li {\bf Small multiples}  or {\bf trellis plots}
are plots consisting of multiple panels in a 
regular grid.
\bi
\li Each panel shows a different subset of the data but all panels use the same type of visualization.
\ei
\li  {\bf Compound figures} consist of separate figure panels assembled in an arbitrary arrangement.
\bi
\li  Each panel shows entirely different visualizations, or possibly even different datasets.
\ei\ei
\end{frame}

\bfr{Trellis plot}
\fig{.8}{titanic-passenger-breakdown-1.png}
\centerline{Titanic passengers}
\end{frame}

\begin{frame}
\fig{.8}{movie-rankings-1.png}
\end{frame}

\begin{frame}
\fig{1}{BA-degrees-variable-y-lims-1.png}
\end{frame}

\begin{frame}
\fig{1}{BA-degrees-fixed-y-lims-1.png}
\end{frame}

\bfr{Arrange panels logically}
\Show{Always arrange the panels in a small multiples plot in a meaningful and logical order.}
\bigskip
\bi
\li In the Titanic plot the rows go from the highest class (1st class) to the lowest class (3rd class).
\li In the movie plot the panels go by increasing years from the top left to the bottom right.
\li  In the college majors plot  the panels go by decreasing average degree popularity, such that the most popular degrees are in the top row and/or to the left and the least popular degrees are in the bottom row and/or to the right.
\ei
\end{frame}

\bfr{Compound figures}
\fig{1}{BA-degrees-compound-1.png}
\bi
\li Always label panels (a, b, etc.)
\li Pay attention to how the panels fit together.
\ei
\end{frame}

\bfr{Bad labels}
\fig{1}{BA-degrees-compound-bad-1.png}
\bi
\li Labels too large and black
\li Labels are wrong font
\li Labelling needs to be consistent throughout a document
\ei
\end{frame}

\bfr{Use a consistent visual language}
\fig{.9}{athletes-composite-inconsistent-1.png}
\scriptsize
\bi\li Colors and left/right for male/female change randomly \ei
\end{frame}


\bfr{Use a consistent visual language}
\fig{.9}{athletes-composite-good-1.png}
\scriptsize
\bi\li Colors and left/right for male/female consistent \ei
\end{frame}


\bfr{Align axes}
\fig{.9}{athletes-composite-misaligned-1.png}

\end{frame}

\end{document}

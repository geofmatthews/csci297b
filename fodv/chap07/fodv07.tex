\documentclass{beamer}
\usetheme{Singapore}
\usepackage{changepage}

%\usepackage{pstricks,pst-node,pst-tree}
\usepackage{amssymb,latexsym}
\usepackage{tikz}
\usepackage{graphicx}
\usepackage{fancyvrb}
\usepackage{hyperref}
\usepackage{fancybox}
\usepackage[listings]{tcolorbox}

\definecolor{codegreen}{rgb}{0,0.6,0}
\definecolor{codegray}{rgb}{0.5,0.5,0.5}
\definecolor{codepurple}{rgb}{0.58,0,0.82}
\definecolor{backcolour}{rgb}{0.95,0.95,0.92}

\lstdefinestyle{mystyle}{
    language=Python,
    backgroundcolor=\color{backcolour},   
    commentstyle=\color{codegreen},
    keywordstyle=\color{magenta},
    numberstyle=\tiny\color{codegray},
    stringstyle=\color{codepurple},
    basicstyle=\ttfamily\scriptsize,
    breakatwhitespace=false,         
    breaklines=true,                 
    captionpos=b,                    
    keepspaces=true,                 
    numbers=left,                    
    numbersep=5pt,                  
    showspaces=false,                
    showstringspaces=false,
    showtabs=false,                  
    tabsize=2,
    escapechar=|,
    frame=single
}

\lstset{style=mystyle}


\newcommand{\bi}{\begin{itemize}}
\newcommand{\li}{\item}
\newcommand{\ei}{\end{itemize}}
\newcommand{\Show}[1]{
\begin{center}
\shadowbox{\begin{minipage}{0.8\textwidth}
          #1
          \end{minipage}}
\end{center}
}
\newcommand{\arrow}{\ensuremath{\rightarrow}}

\newcommand{\uparr}{\ensuremath{\uparrow}}


\newcommand{\fig}[2]{\centerline{\includegraphics[width=#1\textwidth]{#2}}}

\newcommand{\bfr}[1]{\begin{frame}[fragile]\frametitle{{ #1 }}}
\newcommand{\efr}{\end{frame}}

\newcommand{\cola}[1]{\begin{columns}\begin{column}{#1\textwidth}}
\newcommand{\colb}[1]{\end{column}\begin{column}{#1\textwidth}}
\newcommand{\colc}{\end{column}\end{columns}}


\title{Fundamentals of Data Visualization}
\author{Chapter 7}

\begin{document}

\begin{frame}
\maketitle
\end{frame}

\bfr{Visualizing Distributions}
\bi
\li We often want to know how a variable is distributed.
\li For example, we might want to know
how many passengers of what ages there were on the Titanic, i.e., how many children, young adults, middle-aged people, seniors, and so on. 
\li This is called the age {\bf distribution}.
\li A standard visualization is either a histogram or a density plot.
\ei

\end{frame}

\bfr{A single distribution}
\cola{0.2}
\scriptsize
\begin{tabular}{lr}
Age range&	Count\\\hline
0–5	 & 36\\
6–10	 & 	19\\
11–15		 & 18\\
16–20		 & 99\\
21–25		 & 139\\
26–30	 & 	121	\\
31–35	 & 	76\\
36–40	 & 	74\\
41–45	 & 	54\\
46–50	 & 	50\\
51–55		 & 26\\
56–60		 & 22	\\
61–65	 & 	16\\
66–70		 & 3\\
71–75	 & 	3
\end{tabular}
\colb{0.8}
\fig{1}{titanic-ages-hist1-1.png}
\colc
\end{frame}

\bfr{Bin width is crucial}

\fig{1}{titanic-ages-hist-grid-1.png}
\bi\li When making a histogram, always explore multiple bin widths.\ei
\end{frame}

\bfr{Density plots}

\fig{1}{titanic-ages-dens1-1.png}
\scriptsize
\bi
\li Curve is estimated from the data using {\bf kernel density estimation}.
\li We add up all values in a small width (the {\em bandwidth}).
\li Usually weighted by a Gaussian kernel.
\ei
\end{frame}

\bfr{Density curves depend upon kernel and bandwidth}
\fig{1}{titanic-ages-dens-grid-1.png}
\scriptsize

(a) Gaussian kernel, bandwidth = 0.5; (b) Gaussian kernel, bandwidth = 2;\\ (c) Gaussian kernel, bandwidth = 5; (d) Rectangular kernel, bandwidth = 2
\end{frame}

\bfr{Density curves}

\fig{.8}{titanic-ages-dens1-1.png}
\bi
\li Usually scaled so the area sums to one.
\li For the Titanic data, ages range from 0 to 75.
\li The average height should be about $1/75 \approx 0.013$
\ei

\end{frame}

\bfr{Density curves produce data where there is none}

\fig{1}{titanic-ages-dens-negative-1.png}

\end{frame}

\bfr{Histograms vs. Density plots}

\bi
\li Some people are vehemently against density plots and believe that they are arbitrary and misleading.
\li Others realize that histograms can be just as arbitrary and misleading.
\li Density estimates have an inherent advantage over histograms as soon as we want to visualize more than one distribution at a time.
\li Cumulative density and q-q plots plot distributions without the
arbitrary choices, but are difficult to interpret.
\ei
\end{frame}

\bfr{Stacked bar charts for multiple distributions}
\fig{.8}{titanic-age-stacked-hist-1.png}
\scriptsize
\bi
\li Very difficult to interpret.  Where is the zero for women?
\li Were men and women passengers generally of the same age, or was there an age difference between the genders?
\li The bars for women cannot be compared; they do not have a common zero.
\ei
\end{frame}
\scriptsize
\bfr{All bars start at zero, but are transparent}
\fig{0.8}{titanic-age-overlapping-hist-1.png}
\bi
\li Now it appears that there are actually three different groups, not just two.
\li We’re still not entirely sure where each bar starts and ends.
\li A semi-transparent bar drawn on top of another tends to not look like a semi-transparent bar but instead like a bar drawn in a different color.
\ei
\end{frame}


\bfr{Transparent density plots}
\fig{0.8}{titanic-age-overlapping-dens-1.png}
\bi
\li Density plots generally do better with transparency,
because the curves usually don't line up.
\li This dataset is bad for transparency; the curves
are similar up to about age 17.
\ei
\end{frame}

\bfr{Separate density plots}
\fig{1}{titanic-age-fractional-dens-1.png}
\bi
\li Show each subpopulation in relation to the total.
\li Easy to see there were many fewer women at around age 20.
\ei
\end{frame}

\bfr{Age pyramid popular with populations}
\fig{.8}{titanic-age-pyramid-1.png}

\bi
\li Only works for two distributions
\ei

\end{frame}

\bfr{Density plots for multiple distributions work well}
\fig{1}{butterfat-densitites-1.png}

\bi
\li Distributions must be somewhat distinct.
\li Distributions should be contiguous.
\ei

\end{frame}



\end{document}

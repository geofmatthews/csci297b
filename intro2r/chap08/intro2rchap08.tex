\documentclass{beamer}
\usetheme{Singapore}
\usepackage{changepage}
\usepackage[T1]{fontenc}

%\usepackage{pstricks,pst-node,pst-tree}
\usepackage{amssymb,latexsym,dirtree}
\usepackage{tikz}
\usepackage{graphicx}
\usepackage{fancyvrb}
%\usepackage{hyperref}
\usepackage{fancybox}
\usepackage[listings]{tcolorbox}

\definecolor{codegreen}{rgb}{0,0.6,0}
\definecolor{codegray}{rgb}{0.5,0.5,0.5}
\definecolor{codepurple}{rgb}{0.58,0,0.82}
\definecolor{backcolour}{rgb}{0.95,0.95,0.92}

\lstdefinestyle{mystyle}{
    language=Python,
    backgroundcolor=\color{backcolour},   
    commentstyle=\color{codegreen},
    keywordstyle=\color{magenta},
    numberstyle=\tiny\color{codegray},
    stringstyle=\color{codepurple},
    basicstyle=\ttfamily\normalsize,
    breakatwhitespace=false,         
    breaklines=true,                 
    captionpos=b,                    
    keepspaces=true,                 
    numbers=left,                    
    numbersep=5pt,                  
    showspaces=false,                
    showstringspaces=false,
    showtabs=false,                  
    tabsize=2,
    escapechar=|,
    frame=single
}

\lstset{style=mystyle}
\lstset{extendedchars=\true}



\newcommand{\bi}{\begin{itemize}}
\newcommand{\li}{\item}
\newcommand{\ei}{\end{itemize}}
\newcommand{\Show}[1]{
\begin{center}
\shadowbox{\begin{minipage}{0.8\textwidth}
          #1
          \end{minipage}}
\end{center}
}
\newcommand{\arrow}{\ensuremath{\rightarrow}}

\newcommand{\uparr}{\ensuremath{\uparrow}}


\newcommand{\fig}[2]{\centerline{\includegraphics[width=#1\textwidth]{#2}}}
\newcommand{\figg}[2]{\includegraphics[width=#1\textwidth]{#2}}

\newcommand{\bfr}[1]{\begin{frame}[fragile]\frametitle{{ #1 }}}
\newcommand{\efr}{\end{frame}}

\newcommand{\cola}[1]{\begin{columns}\begin{column}{#1\textwidth}}
\newcommand{\colb}[1]{\end{column}\begin{column}{#1\textwidth}}
\newcommand{\colc}{\end{column}\end{columns}}


\title{{https://intro2r.com/} Chapter 8}
\author{CSCI 297b, Spring 2023}

\begin{document}

\begin{frame}
\maketitle
\end{frame}

\bfr{Reproducible reports with R markdown}
\bi
\li Simple and easy plain text language.
\bi\li {\bf way} simpler and easier than HTML\ei
\li Includes formatting, headers, fonts, etc.
\li Includes executable R code
\ei
\end{frame}

\bfr{Why use R markdown?}
\bi
\li Transparency in experimental methodology, observation, collection of data and analytical methods.
\li Public availability and re-usability of scientific data
\li Public accessibility and transparency of scientific communication
\li Using web-based tools to facilitate scientific collaboration
\li Links all your data and your code and your writing in a single document
\li Wide variety of output formats from a single document:\\
{ pdf \hfill html \hfill web pages \hfill MS Word }
\ei
\end{frame}

\bfr{Rmarkdown package}
\begin{verbatim}
library(rmarkdown)
\end{verbatim}

\bi
\li You will need \LaTeX\ if you want to publish PDF
\li Already installed on RStudio Workbench
\ei
\end{frame}

\bfr{Create, save, and knit new Rmarkdown document}\tiny
\begin{verbatim}
---
title: "demo"
output:
  word_document: default
  html_document:
    df_print: paged
---

```{r setup, include=FALSE}
knitr::opts_chunk$set(echo = TRUE)
```

## R Markdown

This is an R Markdown document. Markdown is a simple formatting syntax for authoring HTML, PDF, and MS Word documents. For more details on using R Markdown see <http://rmarkdown.rstudio.com>.

When you click the **Knit** button a document will be generated that includes both content as well as the output of any embedded R code chunks within the document. You can embed an R code chunk like this:

```{r cars}
summary(cars)
```

## Including Plots

You can also embed plots, for example:

```{r pressure, echo=FALSE}
plot(pressure)
```

Note that the `echo = FALSE` parameter was added to the code chunk to prevent printing of the R code that generated the plot.
\end{verbatim}
\end{frame}

\bfr{Rmarkdown anatomy}
\hspace{-0.25in}
\fig{1.2}{rm_components}
\end{frame}

\bfr{YAML header}
\begin{verbatim}
---
title: My first R markdown document
author: Jane Doe
date: March 01, 2020
output: html_document
---
\end{verbatim}
\bi
\li YAML stands for ‘YAML Ain’t Markup Language’.
\li It contains the metadata and options for the entire document such as the author name, date, output format, etc.
\li The YAML header is surrounded before and after by a --- on its own line.
\li In RStudio a minimal YAML header is automatically created for you 
\ei
\end{frame}

\bfr{Use YAML to add table of contents}
\begin{verbatim}
---
title: My first R markdown document
author: Jane Doe
date: March 01, 2020
output:
  html_document:
    toc: yes
---
\end{verbatim}
\end{frame}

\bfr{Formatted text}\scriptsize
\begin{verbatim}
#### Benthic Biodiversity experiment
These data were obtained from a mesocosm experiment which aimed to 
examine the effect of benthic polychaete (*Nereis diversicolor*) biomass 
on sediment nutrient (NH~4~, NO~3~ and PO~3~) release. At the start 
of the experiment replicate mesocosms were filled with 20 cm^2^ of 
**homogenised** marine sediment and assigned to one of five 
polychaete biomass treatments (0, 0.5, 1, 1.5, 2 g per mesocosm).
\end{verbatim}

\fig{1}{rmdknitted}
\end{frame}

\bfr{Markdown cheat sheets}

\bi
\li
\url{https://www.markdownguide.org/cheat-sheet/}
\li
\url{https://www.rstudio.com/wp-content/uploads/2015/02/rmarkdown-cheatsheet.pdf}
\ei
\end{frame}

\bfr{Headings}
\begin{verbatim}
# Benthic Biodiversity experiment
## Benthic Biodiversity experiment
### Benthic Biodiversity experiment
#### Benthic Biodiversity experiment
##### Benthic Biodiversity experiment
###### Benthic Biodiversity experiment
\end{verbatim}
\bi\li
Result in headings in decreasing size order.
\ei
\end{frame}

\bfr{Comments in markdown}
\begin{verbatim}
<!--
this is an example of a comment using R markdown.
-->
\end{verbatim}
\bi
\li Same as HTML
\ei
\end{frame}

\bfr{Bullet lists}
\begin{verbatim}
- item 1
- item 2
   - sub-item 2
   - sub-item 3
- item 3
- item 4
\end{verbatim}
\bi
\li indent with spaces
\li can use $+$ or $-$
\ei
\end{frame}
\bfr{Numbered lists}
\begin{verbatim}
1. item 1
1. item 2
     + sub-item 2
     + sub-item 3
1. item 3
1. item 4
\end{verbatim}
\bi
\li will be numbered sequentially
\li nested list can be numbered or not
\ei
\end{frame}


\bfr{Images}
\begin{verbatim}
![Cute grey kitten](images/Cute_grey_kitten.jpg)
\end{verbatim}
\fig{.8}{Cute_grey_kitten.jpg}
\end{frame}


\bfr{Images with the {\tt knitr} package}
\begin{verbatim}
```{r, echo=FALSE, fig.align='center', out.width='50%'}
library(knitr)
include_graphics("images/Cute_grey_kitten.jpg")
```
\end{verbatim}
\begin{center}
\fig{.5}{Cute_grey_kitten}
\end{center}
\end{frame}


\bfr{Links}
\begin{verbatim}
You can include a text for your
clickable [link](https://www.worldwildlife.org)
\end{verbatim}
\end{frame}



\bfr{R code chunks}
\begin{verbatim}
```{r}
Any valid R code goes here
```
\end{verbatim}
\end{frame}


\bfr{Don't echo the source code}
\begin{verbatim}
```{r, summary-stats, echo=FALSE}
x <- 1:10    # create an x variable
y <- 10:1    # create a y variable
dataf <- data.frame(x = x, y = y)
summary(dataf)
```
\end{verbatim}
\bi
\li {\tt summary-stats} names the code chunk
\ei
\end{frame}


\bfr{Show the code but hide the results}
\begin{verbatim}
```{r, summary-stats, results='hide'}
x <- 1:10    # create an x variable
y <- 10:1    # create a y variable
dataf <- data.frame(x = x, y = y)
summary(dataf)
```
\end{verbatim}
\end{frame}
\bfr{Run but hide both code and results}
\begin{verbatim}
```{r, summary-stats, include=FALSE}
x <- 1:10    # create an x variable
y <- 10:1    # create a y variable
dataf <- data.frame(x = x, y = y)
summary(dataf)
```
\end{verbatim}
\end{frame}


\bfr{plots are shown immediately after the code chunk}
\begin{verbatim}
```{r, simple-plot}
x <- 1:10    # create an x variable
y <- 10:1    # create a y variable
dataf <- data.frame(x = x, y = y)
plot(dataf$x, dataf$y, xlab = "x axis", ylab = "y axis")
```
\end{verbatim}
\bi
\li Can change figure dimensions with {\tt fig.width=} and {\tt fig.height=} (in inches)
\li Can change alignment with {\tt fig.align=} ({\tt `center'} or {\tt `right'})
\li Captions with {\tt fig.cap=}
\li Hide with {\tt fig.show='hide'}
\ei
\end{frame}

\bfr{Tables can be done in markdown}
\begin{verbatim}
    |     x      |    y       |
    |:----------:|:----------:|
    |  1         |   5        |   
    |  2         |   4        |
    |  3         |   3        |
    |  4         |   2        |
    |  5         |   1        |
\end{verbatim}
\end{frame}

\bfr{Tables using {\tt kable} from {\tt knitr}}
\begin{verbatim}
```{r, kable-table}
library(knitr)
kable(iris[1:10,])
```
\end{verbatim}
\end{frame}

\bfr{R inline code}
\begin{verbatim}
The square root of 2 is `r sqrt(2)`
\end{verbatim}
\end{frame}

\bfr{The Definitive Guide}
\bi
\li\url{https://bookdown.org/yihui/rmarkdown/}
\ei
\end{frame}




\end{document}

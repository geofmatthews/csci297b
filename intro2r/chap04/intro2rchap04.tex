\documentclass{beamer}
\usetheme{Singapore}
\usepackage{changepage}
\usepackage[T1]{fontenc}

%\usepackage{pstricks,pst-node,pst-tree}
\usepackage{amssymb,latexsym,dirtree}
\usepackage{tikz}
\usepackage{graphicx}
\usepackage{fancyvrb}
%\usepackage{hyperref}
\usepackage{fancybox}
\usepackage[listings]{tcolorbox}

\definecolor{codegreen}{rgb}{0,0.6,0}
\definecolor{codegray}{rgb}{0.5,0.5,0.5}
\definecolor{codepurple}{rgb}{0.58,0,0.82}
\definecolor{backcolour}{rgb}{0.95,0.95,0.92}

\lstdefinestyle{mystyle}{
    language=Python,
    backgroundcolor=\color{backcolour},   
    commentstyle=\color{codegreen},
    keywordstyle=\color{magenta},
    numberstyle=\tiny\color{codegray},
    stringstyle=\color{codepurple},
    basicstyle=\ttfamily\normalsize,
    breakatwhitespace=false,         
    breaklines=true,                 
    captionpos=b,                    
    keepspaces=true,                 
    numbers=left,                    
    numbersep=5pt,                  
    showspaces=false,                
    showstringspaces=false,
    showtabs=false,                  
    tabsize=2,
    escapechar=|,
    frame=single
}

\lstset{style=mystyle}
\lstset{extendedchars=\true}



\newcommand{\bi}{\begin{itemize}}
\newcommand{\li}{\item}
\newcommand{\ei}{\end{itemize}}
\newcommand{\Show}[1]{
\begin{center}
\shadowbox{\begin{minipage}{0.8\textwidth}
          #1
          \end{minipage}}
\end{center}
}
\newcommand{\arrow}{\ensuremath{\rightarrow}}

\newcommand{\uparr}{\ensuremath{\uparrow}}


\newcommand{\fig}[2]{\centerline{\includegraphics[width=#1\textwidth]{#2}}}

\newcommand{\bfr}[1]{\begin{frame}[fragile]\frametitle{{ #1 }}}
\newcommand{\efr}{\end{frame}}

\newcommand{\cola}{\begin{columns}\begin{column}{0.6\textwidth}}
\newcommand{\colb}{\end{column}\begin{column}{0.4\textwidth}}
\newcommand{\colc}{\end{column}\end{columns}}


\title{{https://intro2r.com/} Chapter 4}
\author{CSCI 297b, Spring 2023}

\begin{document}

\begin{frame}
\maketitle
\end{frame}

\bfr{Base, lattice, and ggplot2 graphics}

\bi
\li Base graphics:  easy, but good style takes work
\li Lattice graphics:  best with complex multi-dimensional data using panel plots
\li Grammar of graphics:  logical development, very good defaults
\ei
\end{frame}

\bfr{Plot panel in RStudio}
\fig{1}{bg_plots1}
\end{frame}

\bfr{Plot panel in RStudio, Zoom button}
\fig{1}{bg_plots2}
\end{frame}

\bfr{Plot panel in RStudio, save button}
\fig{1}{bg_plots3.png}
\end{frame}

\bfr{Scatterplots}
{\scriptsize
\begin{verbatim}
flowers <- read.table(file = 'data/flower.txt', 
                        header = TRUE, sep = "\t", 
                        stringsAsFactors = TRUE)
plot(flowers$weight)
## or
## with(flowers, plot(weight)) 
\end{verbatim}
}
\fig{.8}{plot1-1}
\end{frame}

\bfr{Scatterplots}
{\scriptsize
\begin{verbatim}
plot(x = flowers$weight, y = flowers$shootarea)
## or
## plot(flowers$shootarea ~ flowers$weight)
\end{verbatim}
}
\fig{.8}{plot4-1.png}
\end{frame}

\bfr{Scatterplots}
{\scriptsize
\begin{verbatim}
my_x <- 1:10
my_y <- seq(from = 1, to = 20, by = 2)
par(mfrow = c(2, 2))
plot(my_x, my_y, type = "l")
plot(my_x, my_y, type = "b")
plot(my_x, my_y, type = "o")
plot(my_x, my_y, type = "c")
\end{verbatim}
}
\fig{.6}{plot6-1.png}
\end{frame}

\bfr{Plot}
\bi
\li {\tt plot} has many options 
\li Can add more points, lines, text, {\em etc.}
\li {\tt plot} is a generic function:  it can change its behavior based
on what kind of object it is plotting
\ei
\end{frame}


\bfr{Histograms}
{\scriptsize
\begin{verbatim}
hist(flowers$height)
\end{verbatim}
}
\fig{.8}{plot7-1.png}
\end{frame}


\bfr{Histograms}
{\scriptsize
\begin{verbatim}
brk <- seq(from = 0, to = 18, by = 1)
hist(flowers$height, breaks = brk, main = "petunia height")
\end{verbatim}
}
\fig{.8}{plot8-1.png}
\end{frame}

\bfr{Histograms}
{\scriptsize
\begin{verbatim}
brk <- seq(from = 0, to = 18, by = 1)
hist(flowers$height, breaks = brk, main = "petunia height",
      freq = FALSE)
\end{verbatim}
}
\fig{.8}{plot9-1.png}
\end{frame}

\bfr{Histograms}
{\scriptsize
\begin{verbatim}
dens <- density(flowers$height)
hist(flowers$height, breaks = brk, main = "petunia height",
      freq = FALSE)
lines(dens)
\end{verbatim}
}
\fig{.8}{plot10-1.png}
\end{frame}

\bfr{Do exercise 4 part 1}
\end{frame}

\bfr{Boxplots}
{\scriptsize
\begin{verbatim}
boxplot(flowers$weight, ylab = "weight (g)")
\end{verbatim}
}
\fig{0.8}{plot11-1.png}
\end{frame}

\bfr{Boxplots}
{\scriptsize
\begin{verbatim}
boxplot(weight ~ nitrogen, data = flowers, 
         ylab = "weight (g)", xlab = "nitrogen level")
\end{verbatim}
}
\fig{0.8}{plot12-1.png}
\end{frame}


\bfr{Change order of factor}
{\scriptsize
\begin{verbatim}
flowers$nitrogen <- factor(flowers$nitrogen, 
                    levels = c("low", "medium", "high"))
boxplot(weight ~ nitrogen, data = flowers, 
          ylab = "weight (g)", xlab = "nitrogen level")
\end{verbatim}
}
\fig{0.8}{plot13-1.png}
\end{frame}

\bfr{Dynamite plots must die}
\bi
\li\url{https://simplystatistics.org/posts/2019-02-21-dynamite-plots-must-die/}
\li Use boxplots in combination with dotplots
\ei
\fig{1}{show-data-1.png}
\end{frame}

\bfr{Two factors}
{\scriptsize
\begin{verbatim}
boxplot(weight ~ nitrogen * treat, data = flowers, 
         ylab = "weight (g)", xlab = "nitrogen level")
\end{verbatim}
}
\fig{.8}{plot14-1.png}
\bi\li Some labels missing\ei
\end{frame}

\bfr{Reduce size of label}
{\scriptsize
\begin{verbatim}
boxplot(weight ~ nitrogen * treat, data = flowers, 
         ylab = "weight (g)", xlab = "nitrogen level", 
         cex.axis = 0.7)
\end{verbatim}
}
\fig{.8}{plot15-1.png}
\end{frame}

\bfr{Violin plots}
{\scriptsize
\begin{verbatim}
library(vioplot)
vioplot(weight ~ nitrogen, data = flowers, 
         ylab = "weight (g)", xlab = "nitrogen level",
         col = "lightblue")
\end{verbatim}
}
\bi\li Kernel density estimate on its side\ei
\fig{.8}{plot16-1.png}
\end{frame}

\bfr{Dot charts}
{\scriptsize
\begin{verbatim}
dotchart(flowers$height)
\end{verbatim}
}
\bi
\li $x$ axis is height
\li $y$ axis is order in data frame (bottom to top)
\ei
\fig{.8}{plot17-1.png}
\end{frame}

\bfr{Outliers}
\fig{1}{plot18-1.png}
\end{frame}


\bfr{Grouping data}
{\scriptsize
\begin{verbatim}
dotchart(flowers$height, groups = flowers$nitrogen)
\end{verbatim}
}
\fig{1}{plot19-1.png}
\end{frame}


\bfr{Pairs plots}
{\scriptsize
\begin{verbatim}
pairs(flowers[, c("height", "weight", "leafarea", 
                "shootarea", "flowers")])
# or we could use the equivalent
# pairs(flowers[, 4:8])
\end{verbatim}
}
\fig{.8}{plot20-1.png}
\end{frame}

\bfr{Pairs plots, add a LOWESS fit}
{\scriptsize
\begin{verbatim}
pairs(flowers[, c("height", "weight", "leafarea", 
                "shootarea", "flowers")], 
                 panel = panel.smooth)
\end{verbatim}
}
\fig{.8}{plot21-1.png}
\end{frame}


\bfr{Customized {\tt panel} function for pairs plots}
{\scriptsize
\begin{verbatim}
panel.cor <- function(x, y, digits = 2, prefix = "", 
                      cex.cor, ...)
{
    usr <- par("usr")
    par(usr = c(0, 1, 0, 1))
    r <- abs(cor(x, y))
    txt <- format(c(r, 0.123456789), digits = digits)[1]
    txt <- paste0(prefix, txt)
    if(missing(cex.cor)) cex.cor <- 0.8/strwidth(txt)
    text(0.5, 0.5, txt, cex = cex.cor * r)
}
\end{verbatim}
}
\end{frame}

\bfr{Panel function in use}
{\scriptsize
\begin{verbatim}
pairs(flowers[, c("height", "weight", "leafarea", 
                "shootarea", "flowers")], 
                 lower.panel = panel.cor)
\end{verbatim}
}
\fig{.8}{plot23-1.png}
\end{frame}

\bfr{A panel function for the diagonal}
{\scriptsize
\begin{verbatim}
panel.hist <- function(x, ...)
{
    usr <- par("usr")
    par(usr = c(usr[1:2], 0, 1.5) )
    h <- hist(x, plot = FALSE)
    breaks <- h$breaks; nB <- length(breaks)
    y <- h$counts; y <- y/max(y)
    rect(breaks[-nB], 0, breaks[-1], y, col = "cyan", ...)
}
\end{verbatim}
}
\end{frame}

\bfr{Using 3 panel functions}
{\scriptsize
\begin{verbatim}
pairs(flowers[, c("height", "weight", "leafarea", 
                "shootarea", "flowers")], 
                 lower.panel = panel.cor,
                 diag.panel = panel.hist,
                 upper.panel = panel.smooth)
\end{verbatim}
}
\fig{.8}{plot25-1.png}
\end{frame}

\bfr{Conditional scatterplot}
{\scriptsize
\begin{verbatim}
coplot(flowers ~ weight | leafarea, data = flowers)
\end{verbatim}
}
\fig{1}{plot26-1.png}
\end{frame}

\bfr{Conditional scatterplot, overlap=0}
{\scriptsize
\begin{verbatim}
coplot(flowers ~ weight | leafarea, data = flowers, overlap = 0)
\end{verbatim}
}
\fig{1}{plot27-1.png}
\end{frame}

\bfr{Conditional scatterplot with factors}
{\scriptsize
\begin{verbatim}
coplot(flowers ~ weight | nitrogen, data = flowers)
\end{verbatim}
}
\fig{1}{plot28-1.png}
\end{frame}

\bfr{Conditional scatterplot with two factors}
{\scriptsize
\begin{verbatim}
coplot(flowers ~ weight | nitrogen * treat, data = flowers)
\end{verbatim}
}
\fig{1}{plot29-1.png}
\end{frame}


\bfr{Lattice graphics}
\bi
\li Improved versions of basic graphics
\li Use formula notation
\li Must load library, {\tt library(lattice)}
\ei
\end{frame}

\bfr{Lattice histogram}
{\scriptsize
\begin{verbatim}
library(lattice)
histogram(~ height, type = "count", data = flowers)
\end{verbatim}
}
\fig{.8}{plot31-1.png}
\end{frame}



\bfr{Lattice boxplot (box and whisker plot)}
{\scriptsize
\begin{verbatim}
bwplot(weight ~ nitrogen, data = flowers)
\end{verbatim}
}
\fig{.8}{plot32-1.png}
\end{frame}

\bfr{Some lattice functions and their base R equivalents}
\begin{tabular}{lll}
Graph & Lattice & Base R \\\hline
scatterplot	&xyplot()&	plot()\\
frequency histogram	&histogram(type = "count")	&hist()\\
boxplot&	bwplot()	&boxplot()\\
Cleveland dotplot&	dotplot()&	dotchart()\\
scatterplot matrix&	splom()&	pairs()\\
conditioning plot	&xyplot(y $\sim$ x | z)	&coplot()
\end{tabular}
\end{frame}

\bfr{Lattice graphics with multiple panels}
{\scriptsize
\begin{verbatim}
histogram(~ height | nitrogen, type = "count", data = flowers)
\end{verbatim}
}
\fig{.8}{plot33-1.png}

\end{frame}

\bfr{Lattice graphics with multiple panels}
{\scriptsize
\begin{verbatim}
histogram(~ height | nitrogen, type = "count", 
           layout = c(1, 3), data = flowers)
\end{verbatim}
}
\fig{.8}{plot34-1.png}

\end{frame}

\bfr{Conditional boxplots}
{\scriptsize
\begin{verbatim}
bwplot(weight ~ nitrogen | block, data = flowers)
\end{verbatim}
}
\fig{.8}{plot35-1.png}

\end{frame}

\bfr{Conditional boxplots, block must be factor}
{\scriptsize
\begin{verbatim}
bwplot(weight ~ nitrogen | factor(block), data = flowers)
\end{verbatim}
}
\fig{.8}{plot36-1.png}

\end{frame}


\bfr{Conditional boxplots}
{\scriptsize
\begin{verbatim}
xyplot(height ~ weight | nitrogen * treat, data = flowers)
\end{verbatim}
}
\fig{.8}{plot37-1.png}

\end{frame}

\bfr{Conditional boxplots}
{\scriptsize
\begin{verbatim}
xyplot(flowers ~ shootarea | nitrogen * treat, 
        groups = block, auto.key = TRUE, data = flowers)
\end{verbatim}
}
\fig{.8}{plot38-1.png}

\end{frame}

\bfr{Customizing Base R Graphics}

\bi
\li There are many customizable features.
\li Not all customizations are available for all plots.
\li Refer to text: \url{https://intro2r.com/custom_plot.html}
\ei

\end{frame}

\bfr{Multiple plots with {\tt par(mfrow == ...)}}
{\scriptsize
\begin{verbatim}
par(mfrow = c(1, 2))
plot(flowers$weight, flowers$shootarea, xlab = "weight",
      ylab = "shoot area")
boxplot(shootarea ~ nitrogen, data = flowers, cex.axis = 0.6)
\end{verbatim}
}
\fig{.8}{plot55-1.png}

\end{frame}


\bfr{Multiple plots with {\tt par(mfrow == ...)}}
{\scriptsize
\begin{verbatim}
par(mfrow = c(2, 2))
plot(flowers$weight, flowers$shootarea, xlab = "weight",
      ylab = "shoot area")
boxplot(shootarea ~ nitrogen, cex.axis = 0.8, data = flowers)
hist(flowers$weight, main ="")
dotchart(flowers$weight)
\end{verbatim}
}
\fig{.7}{plot56-1.png}

\end{frame}

\bfr{Multiple plots with {\tt layout()}}
{\scriptsize
\begin{verbatim}
layout_mat <- matrix(c(2, 0, 1, 3), nrow = 2, ncol = 2,
                      byrow = TRUE)
layout_mat
##      [,1] [,2]
## [1,]    2    0
## [2,]    1    3
\end{verbatim}
}

\bi
\li This gives us two rows and two columns
\li The first plot will be lower left
\li The second plot will be upper left
\li The third plot will be lower right
\li Nothing will be upper right
\ei

\end{frame}


\bfr{We can specify row and column sizes {\tt layout()}}
{\scriptsize
\begin{verbatim}
my_lay <- layout(mat = layout_mat, 
                 heights = c(1, 3),
                 widths = c(3, 1), respect =TRUE)
layout.show(my_lay)
\end{verbatim}
}

\fig{0.7}{plot58-1.png}

\end{frame}

\bfr{Adjusting margins generally necessary}
{\scriptsize
\begin{verbatim}
plot(flowers$weight, flowers$shootarea, 
     xlab = "weight (g)", ylab = "shoot area (cm2)")
par(mar = c(0, 4, 0, 0))
boxplot(flowers$weight, horizontal = TRUE, frame = FALSE,
        axes =FALSE)
par(mar = c(4, 0, 0, 0))
boxplot(flowers$shootarea, frame = FALSE, axes = FALSE)
\end{verbatim}
}

\fig{0.7}{plot59-1.png}

\end{frame}

\bfr{Saving plots}
\bi
\li Can use RStudio buttons
\li Can also save in code with {\tt pdf, png, jpeg, tiff, bmp}
\ei

{\scriptsize
\begin{verbatim}
pdf("output/myplot.pdf")
plot(...)
dev.off()

png("output/myplot.png")
plot(...)
dev.off()
\end{verbatim}
}

\end{frame}

\bfr{Do exercise 4 part 2}
\end{frame}

\end{document}

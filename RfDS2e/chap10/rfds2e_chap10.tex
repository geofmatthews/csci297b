\documentclass{beamer}
\usetheme{Singapore}
\usepackage{changepage}
\usepackage[T1]{fontenc}
\usepackage[utf8]{inputenc}
%\usepackage{pstricks,pst-node,pst-tree}
\usepackage{amssymb,latexsym,dirtree}
\usepackage{tikz}
\usepackage{graphicx}
\usepackage{fancyvrb}
%\usepackage{hyperref}
\usepackage{fancybox}

\newcommand{\bi}{\begin{itemize}}
\newcommand{\li}{\item}
\newcommand{\ei}{\end{itemize}}
\newcommand{\Show}[1]{
\begin{center}
\shadowbox{\begin{minipage}{0.8\textwidth}
          #1
          \end{minipage}}
\end{center}
}
\newcommand{\arrow}{\ensuremath{\rightarrow}}

\newcommand{\uparr}{\ensuremath{\uparrow}}


\newcommand{\fig}[2]{\centerline{\includegraphics[width=#1\textwidth]{#2}}}
\newcommand{\figg}[2]{\includegraphics[width=#1\textwidth]{#2}}

\newcommand{\bfr}[1]{\begin{frame}[fragile]\frametitle{{ #1 }}}
\newcommand{\efr}{\end{frame}}

\newcommand{\cola}[1]{\begin{columns}\begin{column}{#1\textwidth}}
\newcommand{\colb}[1]{\end{column}\begin{column}{#1\textwidth}}
\newcommand{\colc}{\end{column}\end{columns}}


\title{{https://r4ds.hadley.nz/} Chapter 10\\Layers}
\author{CSCI 297b, Spring 2023}

\begin{document}
\begin{frame}
\maketitle
\end{frame}

\bfr{Aesthetic mappings}

\begin{quotation}
“The greatest value of a picture is when it forces us to notice what we never expected to see.” — John Tukey
\end{quotation}
\end{frame}

\bfr{tidyverse}

\begin{verbatim}
library(tidyverse}
\end{verbatim}
\end{frame}



\bfr{The {\tt mpg} dataset}\scriptsize
\begin{verbatim}
mpg
#> # A tibble: 234 × 11
#>   manufacturer model displ  year   cyl trans      drv     cty   hwy fl   
#>   <chr>        <chr> <dbl> <int> <int> <chr>      <chr> <int> <int> <chr>
#> 1 audi         a4      1.8  1999     4 auto(l5)   f        18    29 p    
#> 2 audi         a4      1.8  1999     4 manual(m5) f        21    29 p    
#> 3 audi         a4      2    2008     4 manual(m6) f        20    31 p    
#> 4 audi         a4      2    2008     4 auto(av)   f        21    30 p    
#> 5 audi         a4      2.8  1999     6 auto(l5)   f        16    26 p    
#> 6 audi         a4      2.8  1999     6 manual(m5) f        18    26 p    
#> # i 228 more rows
#> # i 1 more variable: class <chr>
\end{verbatim}
\bi
\li displ: A car’s engine size, in liters. A numerical variable.
\li
hwy: A car’s fuel efficiency on the highway, in miles per gallon (mpg). A car with a low fuel efficiency consumes more fuel than a car with a high fuel efficiency when they travel the same distance. A numerical variable.
\li
class: Type of car. A categorical variable.
\ei
\end{frame}

\bfr{Groups can go unplotted}\scriptsize
\begin{verbatim}
# Left
ggplot(mpg, aes(x = displ, y = hwy, color = class)) +
  geom_point()

# Right
ggplot(mpg, aes(x = displ, y = hwy, shape = class)) +
  geom_point()
#> Warning: The shape palette can deal with a maximum of 6 discrete values
#> because more than 6 becomes difficult to discriminate; you have 7.
#> Consider specifying shapes manually if you must have them.
#> Warning: Removed 62 rows containing missing values (`geom_point()`).
\end{verbatim}
\figg{.5}{unnamed-chunk-4-1.png}\figg{.5}{unnamed-chunk-4-2.png}
\end{frame}

\bfr{not advised}\scriptsize
\begin{verbatim}
# Left
ggplot(mpg, aes(x = displ, y = hwy, size = class)) +  geom_point()
#> Warning: Using size for a discrete variable is not advised.

# Right
ggplot(mpg, aes(x = displ, y = hwy, alpha = class)) +  geom_point()
#> Warning: Using alpha for a discrete variable is not advised.
\end{verbatim}
\figg{.5}{unnamed-chunk-5-1.png}\figg{.5}{unnamed-chunk-5-2.png}

\bi
\li Implies an order that does not exist.
\ei
\end{frame}

\bfr{ggplot2 defaults}
\bi
\li  It selects a reasonable scale to use with the aesthetic.
\li It constructs a legend that explains the mapping between levels and values. 
\li For x and y aesthetics, ggplot2 does not create a legend.
\li But it creates an axis line with tick marks and a label. 
\li The axis line provides the same information as a legend.
\li It explains the mapping between locations and values.
\ei
\end{frame}

\bfr{Set visual properties manually}
\begin{verbatim}
ggplot(mpg, aes(x = displ, y = hwy)) + 
  geom_point(color = "blue")

\end{verbatim}
\fig{.8}{unnamed-chunk-6-1.png}
\end{frame}

\bfr{Properties of points}
\cola{0.7}
\bi
\li Color as character string,\\ e.g. {\tt color = "blue"}
\li Size in mm, e.g. {\tt size = 1}
\li Shape as integer, e.g. {\tt shape = 1}
\ei
\colb{0.3}
\hspace{-.5in}\fig{1.5}{pointshapes}
\colc
\scriptsize
\bi
\li
R has 25 built-in shapes that are identified by numbers. 
\li There are some seeming duplicates: for example, 0, 15, and 22 are all squares. 
\li The difference comes from the interaction of the {\tt color} and {\tt fill} aesthetics.
\li The hollow shapes (0–14) have a border determined by {\tt color}
\li The solid shapes (15–20) are filled with {\tt color}
\li The filled shapes (21–24) have a border of {\tt color} and are filled with {\tt fill}.
\ei

\vfill
\url{https://ggplot2.tidyverse.org/articles/ggplot2-specs.html}

\end{frame}

\bfr{Do exercise 8}
\end{frame}

\bfr{How are these plots similar?}
\figg{0.5}{unnamed-chunk-9-1.png}\figg{0.5}{unnamed-chunk-9-2.png}
\scriptsize
\begin{verbatim}
# Left
ggplot(mpg, aes(x = displ, y = hwy)) + 
  geom_point()

# Right
ggplot(mpg, aes(x = displ, y = hwy)) + 
  geom_smooth()
#> `geom_smooth()` using method = 'loess' and formula = 'y ~ x'
\end{verbatim}
\end{frame}

\bfr{Mapping arguments}
\bi
\li Every geom function in \verb|ggplot2| takes a mapping argument.
\li It is either defined locally in the geom layer or globally in the \verb|ggplot()| layer.
\li Not every aesthetic works with every geom. 
\li You could set the shape of a point, but you couldn’t set the “shape” of a line. 
\li If you try, \verb|ggplot2| will silently ignore that aesthetic mapping.
\li On the other hand, you could set the linetype of a line. 
\li \verb|geom_smooth()| will draw a different line, with a different linetype, for each unique value of the variable that you map to linetype.
\ei
\end{frame}

\bfr{Mapping arguments}
\begin{verbatim}
# Left
ggplot(mpg, aes(x = displ, y = hwy, shape = drv)) + 
  geom_smooth()

# Right
ggplot(mpg, aes(x = displ, y = hwy, linetype = drv)) + 
  geom_smooth()
\end{verbatim}
\figg{0.5}{unnamed-chunk-11-1.png}\figg{0.5}{unnamed-chunk-11-2.png}
\end{frame}

\bfr{Mapping arguments}\scriptsize
\begin{verbatim}
ggplot(mpg, aes(x = displ, y = hwy, color = drv)) + 
  geom_point() +
  geom_smooth(aes(linetype = drv))
\end{verbatim}
\fig{.8}{unnamed-chunk-12-1.png}

\end{frame}

\bfr{{\tt group} aesthetic}
\bi
\li Many geoms, like \verb|geom_smooth()|, use a single geometric object to display multiple rows of data. 
\li For these geoms, you can set the group aesthetic to a categorical variable to draw multiple objects.
\li \verb|ggplot2| will draw a separate object for each unique value of the grouping variable.
\li In practice, \verb|ggplot2| will automatically group the data for these geoms whenever you map an aesthetic to a discrete variable (as in the linetype example).
\li It is convenient to rely on this feature because the group aesthetic by itself does not add a legend or distinguishing features to the geoms.
\ei
\end{frame}

\bfr{{\tt group} aesthetic}\scriptsize
\begin{verbatim}
# Left
ggplot(mpg, aes(x = displ, y = hwy)) +
  geom_smooth()

# Middle
ggplot(mpg, aes(x = displ, y = hwy)) +
  geom_smooth(aes(group = drv))

# Right
ggplot(mpg, aes(x = displ, y = hwy)) +
  geom_smooth(aes(color = drv), show.legend = FALSE)
\end{verbatim}
\figg{0.33}{unnamed-chunk-13-1.png}\figg{0.33}{unnamed-chunk-13-2.png}\figg{0.33}{unnamed-chunk-13-3.png}
\end{frame}

\bfr{Different aesthetics in different layers}\scriptsize
\begin{verbatim}
ggplot(mpg, aes(x = displ, y = hwy)) + 
  geom_point(aes(color = class)) + 
  geom_smooth()
\end{verbatim}
\fig{.8}{unnamed-chunk-14-1.png}

\end{frame}

\bfr{Different data in different layers}\scriptsize
\begin{verbatim}
ggplot(mpg, aes(x = displ, y = hwy)) + 
  geom_point() + 
  geom_point(
    data = mpg |> filter(class == "2seater"), 
    color = "red"
  ) +
  geom_point(
    data = mpg |> filter(class == "2seater"), 
    shape = "circle open", size = 3, color = "red"
  )
\end{verbatim}
\fig{.65}{unnamed-chunk-15-1.png}

\end{frame}


\bfr{{\tt geom}s change everything}\scriptsize
\begin{verbatim}
# Left
ggplot(mpg, aes(x = hwy)) +
  geom_histogram(binwidth = 2)

# Middle
ggplot(mpg, aes(x = hwy)) +
  geom_density()

# Right
ggplot(mpg, aes(x = hwy)) +
  geom_boxplot()
\end{verbatim}
\figg{0.33}{unnamed-chunk-16-1.png}\figg{0.33}{unnamed-chunk-16-2.png}\figg{0.33}{unnamed-chunk-16-3.png}
\end{frame}

\bfr{Extension packages}
\bi
\li \verb|ggplot2| provides more than 40 geoms.
\li But these don’t cover all possible plots one could make.
\li If you need a different geom, we recommend looking into extension packages first to see if someone else has already implemented it.
\li \url{https://exts.ggplot2.tidyverse.org/gallery/}
\li For example, the \verb|ggridges| package \url{https://wilkelab.org/ggridges} is useful for making ridgeline plots, which can be useful for visualizing the density of a numerical variable for different levels of a categorical variable.
\ei
\end{frame}

\bfr{Ridges example}\scriptsize
\begin{verbatim}
library(ggridges)

ggplot(mpg, aes(x = hwy, y = drv, fill = drv, color = drv)) +
  geom_density_ridges(alpha = 0.5, show.legend = FALSE)
#> Picking joint bandwidth of 1.28
\end{verbatim}
\fig{1}{unnamed-chunk-17-1.png}
\end{frame}

\bfr{The {\tt tidyverse} reference}

\large\url{https://ggplot2.tidyverse.org/reference}
\end{frame}

\bfr{Do exercise 9}
\end{frame}

\bfr{\tt facet\_wrap}\scriptsize
\begin{verbatim}
ggplot(mpg, aes(x = displ, y = hwy)) + 
  geom_point() + 
  facet_wrap(~cyl)
\end{verbatim}
\fig{.8}{unnamed-chunk-20-1.png}
\end{frame}

\bfr{\tt facet\_grid}\scriptsize
\begin{verbatim}
ggplot(mpg, aes(x = displ, y = hwy)) + 
  geom_point() + 
  facet_grid(drv ~ cyl)
\end{verbatim}
\fig{.8}{unnamed-chunk-21-1.png}
\end{frame}


\bfr{Free the scales!}\scriptsize
\begin{verbatim}
ggplot(mpg, aes(x = displ, y = hwy)) + 
  geom_point() + 
  facet_grid(drv ~ cyl, scales = "free_y")
\end{verbatim}
\fig{.8}{unnamed-chunk-22-1.png}
\end{frame}

\bfr{Do exercise 10}
\end{frame}

\bfr{{\tt diamonds} dataset}
\bi
\li The diamonds dataset is in the\verb| ggplot2| package.
\li It contains information on 
$\approx 54,000$ diamonds.
\li It includes  carat, color, clarity, and cut of each diamond. 
\ei
\end{frame}

\bfr{Bar chart}
\begin{verbatim}
ggplot(diamonds, aes(x = cut)) + 
  geom_bar()
\end{verbatim}
\fig{.8}{unnamed-chunk-28-1.png}
\bi
\li \verb|count| is not in the dataset!
\ei
\end{frame}

\bfr{Statistical transformations}
\bi
\li  Many graphs, like scatterplots, plot the raw values of your dataset.
\li Other graphs, like bar charts, calculate new values to plot:
\bi
\li
Bar charts, histograms, and frequency polygons bin your data and then plot bin counts, the number of points that fall in each bin.
\li
Smoothers fit a model to your data and then plot predictions from the model.
\li
Boxplots compute the five-number summary of the distribution and then display that summary as a specially formatted box.
\ei
\li
The algorithm used to calculate new values for a graph is called a {\bf stat}, short for statistical transformation.
\ei
\end{frame}

\bfr{Computing stats}
\fig{1.25}{visualization-stat-bar.png}
\end{frame}

\bfr{Stats}
\bi
\li You can learn which stat a geom uses by inspecting the default value for the \verb|stat| argument.
\li For example, \verb|?geom_bar| shows that the default value for stat is “count”, which means that \verb|geom_bar()| uses \verb|stat_count()|.
\li  \verb|stat_count()| is documented on the same page as \verb|geom_bar()|.
\li If you scroll down, the section called “Computed variables” explains that it computes two new variables: \verb|count| and \verb|prop|.
\ei
\end{frame}

\bfr{Stats}
\bi
\li Every geom has a default stat; and every stat has a default geom. 
\li This means that you can typically use geoms without worrying about the underlying statistical transformation.
\li However, there are  reasons why you might need to use a stat explicitly.

\ei
\end{frame}


\bfr{We might already have the value computed}
\begin{verbatim}
> diamonds |>
+   count(cut) 
# A tibble: 5 × 2
  cut           n
  <ord>     <int>
1 Fair       1610
2 Good       4906
3 Very Good 12082
4 Premium   13791
5 Ideal     21551
\end{verbatim}
\end{frame}



\bfr{We might already have the value computed}\scriptsize
\begin{verbatim}
diamonds |>
  count(cut) |>
  ggplot(aes(x = cut, y = n)) +
  geom_bar(stat = "identity")
\end{verbatim}
\fig{.8}{unnamed-chunk-30-1.png}
\end{frame}



\bfr{We might want to show proportion instead of count}\scriptsize
\begin{verbatim}
ggplot(diamonds, aes(x = cut, y = after_stat(prop), group = 1)) + 
  geom_bar()
\end{verbatim}
\fig{.8}{unnamed-chunk-31-1.png}
\end{frame}



\bfr{We might want to use {\tt stat\_summary}}\scriptsize
\begin{verbatim}
ggplot(diamonds) + 
  stat_summary(
    aes(x = cut, y = depth),
    fun.min = min,
    fun.max = max,
    fun = median
  )
\end{verbatim}
\fig{.8}{unnamed-chunk-32-1.png}
\end{frame}

\bfr{Do exercise 11}
\end{frame}


\bfr{Bar charts can use either {\tt color} or {\tt fill}}\scriptsize
\begin{verbatim}
# Left
ggplot(mpg, aes(x = drv, color = drv)) + 
  geom_bar()

# Right
ggplot(mpg, aes(x = drv, fill = drv)) + 
  geom_bar()
\end{verbatim}
\figg{.5}{unnamed-chunk-34-1.png}\figg{.5}{unnamed-chunk-34-2.png}
\end{frame}



\bfr{Map {\tt fill} to another variable}\scriptsize
\begin{verbatim}
ggplot(mpg, aes(x = drv, fill = class)) + 
  geom_bar()
\end{verbatim}
\fig{.7}{unnamed-chunk-35-1.png}
\bi
\li The stacking is performed automatically using the {\bf position adjustment} specified by the 
\verb|position| argument.
\li Can be "identity", "dodge", or "fill".
\ei
\end{frame}


\bfr{{\tt identity} not useful for bars}\scriptsize
\begin{verbatim}
# Left
ggplot(mpg, aes(x = drv, fill = class)) + 
  geom_bar(alpha = 1/5, position = "identity")

# Right
ggplot(mpg, aes(x = drv, color = class)) + 
  geom_bar(fill = NA, position = "identity")
\end{verbatim}
\figg{.5}{unnamed-chunk-36-1.png}\figg{.5}{unnamed-chunk-36-2.png}
\end{frame}


\bfr{{\tt fill} and {\tt dodge}}\scriptsize
\begin{verbatim}
# Left
ggplot(mpg, aes(x = drv, fill = class)) + 
  geom_bar(position = "fill")

# Right
ggplot(mpg, aes(x = drv, fill = class)) + 
  geom_bar(position = "dodge")
\end{verbatim}
\figg{.5}{unnamed-chunk-37-1.png}\figg{.5}{unnamed-chunk-37-2.png}
\end{frame}


\bfr{{\tt position = jitter} useful for scatterplots}\scriptsize
\begin{verbatim}
ggplot(mpg, aes(x = displ, y = hwy)) + 
  geom_point(position = "jitter")
\end{verbatim}
\fig{.8}{unnamed-chunk-39-1.png}
\bi
\li \verb|geom_jitter()| is shorthand for \verb|geom_point(position = "jitter")|
\ei
\end{frame}

\bfr{Do exercise 12}
\end{frame}

\bfr{\tt coord\_quickmap}\scriptsize
\begin{verbatim}
nz <- map_data("nz")

ggplot(nz, aes(x = long, y = lat, group = group)) +
  geom_polygon(fill = "white", color = "black")

ggplot(nz, aes(x = long, y = lat, group = group)) +
  geom_polygon(fill = "white", color = "black") +
  coord_quickmap()
\end{verbatim}
\figg{.5}{unnamed-chunk-42-1.png}\figg{.5}{unnamed-chunk-42-2.png}
\vfill
\url{https://ggplot2-book.org/maps.html}
\end{frame}


\bfr{Polar coordinates}\scriptsize
\begin{verbatim}
bar <- ggplot(data = diamonds) + 
  geom_bar(
    mapping = aes(x = clarity, fill = clarity), 
    show.legend = FALSE,
    width = 1
  ) + 
  theme(aspect.ratio = 1)
bar + coord_flip()
bar + coord_polar()
\end{verbatim}
\figg{.45}{unnamed-chunk-43-1.png}\figg{.45}{unnamed-chunk-43-2.png}
\end{frame}


\bfr{ggplot2 template}\scriptsize
\begin{verbatim}
ggplot(data = <DATA>) + 
  <GEOM_FUNCTION>(
     mapping = aes(<MAPPINGS>),
     stat = <STAT>, 
     position = <POSITION>
  ) +
  <COORDINATE_FUNCTION> +
  <FACET_FUNCTION>
\end{verbatim}
\fig{1.1}{visualization-grammar.png}
\bi
\li Any graph can be made with these seven parameters.
\li \url{https://vita.had.co.nz/papers/layered-grammar.pdf}
\ei
\end{frame}





\end{document}

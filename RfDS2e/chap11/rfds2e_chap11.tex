\documentclass{beamer}
\usetheme{Singapore}
\usepackage{changepage}
\usepackage[T1]{fontenc}
\usepackage[utf8]{inputenc}
%\usepackage{pstricks,pst-node,pst-tree}
\usepackage{amssymb,latexsym,dirtree}
\usepackage{tikz}
\usepackage{graphicx}
\usepackage{fancyvrb}
%\usepackage{hyperref}
\usepackage{fancybox}

\newcommand{\bi}{\begin{itemize}}
\newcommand{\li}{\item}
\newcommand{\ei}{\end{itemize}}
\newcommand{\Show}[1]{
\begin{center}
\shadowbox{\begin{minipage}{0.8\textwidth}
          #1
          \end{minipage}}
\end{center}
}
\newcommand{\arrow}{\ensuremath{\rightarrow}}

\newcommand{\uparr}{\ensuremath{\uparrow}}


\newcommand{\fig}[2]{\centerline{\includegraphics[width=#1\textwidth]{#2}}}
\newcommand{\figg}[2]{\includegraphics[width=#1\textwidth]{#2}}

\newcommand{\bfr}[1]{\begin{frame}[fragile]\frametitle{{ #1 }}}
\newcommand{\efr}{\end{frame}}

\newcommand{\cola}[1]{\begin{columns}\begin{column}{#1\textwidth}}
\newcommand{\colb}[1]{\end{column}\begin{column}{#1\textwidth}}
\newcommand{\colc}{\end{column}\end{columns}}


\title{{https://r4ds.hadley.nz/} Chapter 10\\Layers}
\author{CSCI 297b, Spring 2023}

\begin{document}

\bfr{Exploratory data analysis}

\bi
\li Generate questions about your data.
\li
Search for answers by visualizing, transforming, and modelling your data.
\li
Use what you learn to refine your questions and/or generate new questions.
\ei
\end{frame}

\bfr{Exploratory data analysis}
\begin{quotation}
“There are no routine statistical questions, only questionable statistical routines.”\\\hfill — Sir David Cox
\end{quotation}
\bigskip

\begin{quotation}
“Far better an approximate answer to the right question, which is often vague, than an exact answer to the wrong question, which can always be made precise.” \\\hfill— John Tukey
\end{quotation}

\end{frame}

\bfr{Exploratory data analysis}
\Show{The key to asking {\em quality} questions is to generate a large {\em quantity} of questions.}
\end{frame}

\bfr{The basic questions}
\bi
\li
What type of variation occurs within my variables?
\li
What type of covariation occurs between my variables?
\ei
\end{frame}

\bfr{\tt diamonds}\scriptsize
\begin{verbatim}
ggplot(diamonds, aes(x = carat)) +
  geom_histogram(binwidth = 0.5)
\end{verbatim}
\fig{.7}{unnamed-chunk-3-1.png}
\bi
\li Which values are the most common? Why?
\li
Which values are rare? Why? Does that match your expectations?
\li
Can you see any unusual patterns? What might explain them?
\ei
\end{frame}


\bfr{\tt diamonds}\scriptsize
\begin{verbatim}
smaller <- diamonds |> 
  filter(carat < 3)

ggplot(smaller, aes(x = carat)) +
  geom_histogram(binwidth = 0.01)
\end{verbatim}
\fig{.7}{unnamed-chunk-4-1.png}
\bi
\li Why are there more diamonds at whole carats and common fractions of carats?
\li
Why are there more diamonds slightly to the right of each peak than there are slightly to the left of each peak?
\ei
\end{frame}

\bfr{Understand subgroups}
\bi
\li How are the observations within each subgroup similar to each other?
\li
How are the observations in separate clusters different from each other?
\li
How can you explain or describe the clusters?
\li
Why might the appearance of clusters be misleading?
\ei

\bigskip
Some of these questions can be answered with the data while some will require domain expertise about the data.
\end{frame}


\bfr{Outliers}\scriptsize
\begin{verbatim}
ggplot(diamonds, aes(x = y)) + 
  geom_histogram(binwidth = 0.5)
\end{verbatim}
\fig{.8}{unnamed-chunk-5-1.png}
\end{frame}


\bfr{Zoom in to see small boxes}\scriptsize
\begin{verbatim}
ggplot(diamonds, aes(x = y)) + 
  geom_histogram(binwidth = 0.5) +
  coord_cartesian(ylim = c(0, 50))
\end{verbatim}
\fig{.8}{unnamed-chunk-6-1.png}
\end{frame}


\bfr{Examine the outlier data}\scriptsize
\begin{verbatim}
unusual <- diamonds |> 
  filter(y < 3 | y > 20) |> 
  select(price, x, y, z) |>
  arrange(y)
unusual
#> # A tibble: 9 × 4
#>   price     x     y     z
#>   <int> <dbl> <dbl> <dbl>
#> 1  5139  0      0    0   
#> 2  6381  0      0    0   
#> 3 12800  0      0    0   
#> 4 15686  0      0    0   
#> 5 18034  0      0    0   
#> 6  2130  0      0    0   
#> 7  2130  0      0    0   
#> 8  2075  5.15  31.8  5.12
#> 9 12210  8.09  58.9  8.06
\end{verbatim}
\bi
\li We know dimensions can't be zero.
\li 31.8 and 58.9 are highly suspicious.
\li We might want to replace all these with NAs.
\ei
\end{frame}


\bfr{Options for suspicious data}
\bi
\li Drop the entire row
\begin{verbatim}
diamonds2 <- diamonds |> 
  filter(between(y, 3, 20))
\end{verbatim}
\li Replace with NAs
\begin{verbatim}
diamonds2 <- diamonds |> 
  mutate(y = if_else(y < 3 | y > 20, NA, y))
\end{verbatim}
\ei
\end{frame}





\bfr{Covariation between categorical and a numerical variable}\scriptsize
\begin{verbatim}
ggplot(diamonds, aes(x = price)) + 
  geom_freqpoly(aes(color = cut), binwidth = 500, linewidth = 0.75)
\end{verbatim}
\fig{.8}{unnamed-chunk-15-1.png}
\end{frame}



\bfr{Covariation between categorical and a numerical variable}\scriptsize
\begin{verbatim}
ggplot(diamonds, aes(x = price, y = after_stat(density))) + 
  geom_freqpoly(aes(color = cut), binwidth = 500, linewidth = 0.75)
\end{verbatim}
\fig{.8}{unnamed-chunk-16-1.png}
\end{frame}





\bfr{Reorder factors}\scriptsize
\begin{verbatim}
ggplot(mpg, aes(x = class, y = hwy)) +
  geom_boxplot()
\end{verbatim}
\fig{.8}{unnamed-chunk-18-1.png}
\end{frame}



\bfr{Reorder factors}\scriptsize
\begin{verbatim}
ggplot(mpg, aes(x = fct_reorder(class, hwy, median), y = hwy)) +
  geom_boxplot()
\end{verbatim}
\fig{.8}{unnamed-chunk-19-1.png}
\end{frame}





\bfr{Long names}\scriptsize
\begin{verbatim}
ggplot(mpg, aes(x = hwy, y = fct_reorder(class, hwy, median))) +
  geom_boxplot()
\end{verbatim}
\fig{.8}{unnamed-chunk-20-1.png}
\end{frame}





\bfr{Two categorical variables}\scriptsize
\begin{verbatim}
ggplot(diamonds, aes(x = cut, y = color)) +
  geom_count()
\end{verbatim}
\fig{.8}{unnamed-chunk-21-1.png}
\end{frame}





\bfr{\tt count}\scriptsize
\cola{0.4}
\begin{verbatim}
diamonds |> 
  count(color, cut)
#> # A tibble: 35 × 3
#>   color cut           n
#>   <ord> <ord>     <int>
#> 1 D     Fair        163
#> 2 D     Good        662
#> 3 D     Very Good  1513
#> 4 D     Premium    1603
#> 5 D     Ideal      2834
#> 6 E     Fair        224
#> # i 29 more rows

diamonds |> 
  count(color, cut) |>  
  ggplot(aes(x = color, y = cut)) +
  geom_tile(aes(fill = n))
\end{verbatim}
\colb{.6}
\fig{1.1}{unnamed-chunk-23-1.png}
\colc
\end{frame}



\bfr{Two numerical variables}\scriptsize
\begin{verbatim}
ggplot(smaller, aes(x = carat, y = price)) +
  geom_point()
\end{verbatim}
\fig{.8}{unnamed-chunk-24-1.png}
\end{frame}



\bfr{{\tt geom\_bin2d} and {\tt geom\_hex}}\scriptsize
\begin{verbatim}
ggplot(smaller, aes(x = carat, y = price)) +
  geom_bin2d()

# install.packages("hexbin")
ggplot(smaller, aes(x = carat, y = price)) +
  geom_hex()
\end{verbatim}
\figg{.5}{unnamed-chunk-26-1.png}\figg{.5}{unnamed-chunk-26-2.png}
\end{frame}



\bfr{Bin one continuous variable}\scriptsize
\begin{verbatim}
ggplot(smaller, aes(x = carat, y = price)) + 
  geom_boxplot(aes(group = cut_width(carat, 0.1)))
\end{verbatim}
\fig{.8}{unnamed-chunk-27-1.png}
\end{frame}

\bfr{Patterns in data}
\bi
\li Could this pattern be due to coincidence (i.e. random chance)?
\li
How can you describe the relationship implied by the pattern?
\li
How strong is the relationship implied by the pattern?
\li
What other variables might affect the relationship?
\li
Does the relationship change if you look at individual subgroups of the data?
\ei
\end{frame}








\end{document}

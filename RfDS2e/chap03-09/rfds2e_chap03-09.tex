\documentclass{beamer}
\usetheme{Singapore}
\usepackage{changepage}
\usepackage[T1]{fontenc}
\usepackage[utf8]{inputenc}
%\usepackage{pstricks,pst-node,pst-tree}
\usepackage{amssymb,latexsym,dirtree}
\usepackage{tikz}
\usepackage{graphicx}
\usepackage{fancyvrb}
%\usepackage{hyperref}
\usepackage{fancybox}

\newcommand{\bi}{\begin{itemize}}
\newcommand{\li}{\item}
\newcommand{\ei}{\end{itemize}}
\newcommand{\Show}[1]{
\begin{center}
\shadowbox{\begin{minipage}{0.8\textwidth}
          #1
          \end{minipage}}
\end{center}
}
\newcommand{\arrow}{\ensuremath{\rightarrow}}

\newcommand{\uparr}{\ensuremath{\uparrow}}


\newcommand{\fig}[2]{\centerline{\includegraphics[width=#1\textwidth]{#2}}}
\newcommand{\figg}[2]{\includegraphics[width=#1\textwidth]{#2}}

\newcommand{\bfr}[1]{\begin{frame}[fragile]\frametitle{{ #1 }}}
\newcommand{\efr}{\end{frame}}

\newcommand{\cola}[1]{\begin{columns}\begin{column}{#1\textwidth}}
\newcommand{\colb}[1]{\end{column}\begin{column}{#1\textwidth}}
\newcommand{\colc}{\end{column}\end{columns}}


\title{{https://r4ds.hadley.nz/} Chapter 3-9}
\author{CSCI 297b, Spring 2023}

\begin{document}

\begin{frame}
\maketitle
\end{frame}

\bfr{The Big Picture}
\fig{1}{whole-game.png}
\end{frame}

\bfr{The {\tt dplyr} package and the {\tt nycflights13} dataset}

\begin{verbatim}
library(nycflights13)
library(tidyverse)
\end{verbatim}
\end{frame}

\bfr{the {\tt nycflights13} dataset}\scriptsize
\begin{verbatim}
> glimpse(flights)
Rows: 336,776
Columns: 19
$ year           <int> 2013, 2013, 2013, 2013,…
$ month          <int> 1, 1, 1, 1, 1, 1, 1, 1,…
$ day            <int> 1, 1, 1, 1, 1, 1, 1, 1,…
$ dep_time       <int> 517, 533, 542, 544, 554…
$ sched_dep_time <int> 515, 529, 540, 545, 600…
$ dep_delay      <dbl> 2, 4, 2, -1, -6, -4, -5…
$ arr_time       <int> 830, 850, 923, 1004, 81…
$ sched_arr_time <int> 819, 830, 850, 1022, 83…
$ arr_delay      <dbl> 11, 20, 33, -18, -25, 1…
$ carrier        <chr> "UA", "UA", "AA", "B6",…
$ flight         <int> 1545, 1714, 1141, 725, …
$ tailnum        <chr> "N14228", "N24211", "N6…
$ origin         <chr> "EWR", "LGA", "JFK", "J…
$ dest           <chr> "IAH", "IAH", "MIA", "B…
$ air_time       <dbl> 227, 227, 160, 183, 116…
$ distance       <dbl> 1400, 1416, 1089, 1576,…
$ hour           <dbl> 5, 5, 5, 5, 6, 5, 6, 6,…
$ minute         <dbl> 15, 29, 40, 45, 0, 58, …
$ time_hour      <dttm> 2013-01-01 05:00:00, 2…
\end{verbatim}
\end{frame}

\bfr{The {\tt dplyr} package}
\bi
\li The first argument is always a data frame.
\li
The subsequent arguments typically describe which columns to operate on, using the variable names (without quotes).
\li
The output is always a new data frame.
\li Each verb operates on either
\bi
\li rows,
\li columns,
\li groups, or 
\li tables
\ei
\ei
\end{frame}

\bfr{The pipe}
\begin{center}
\begin{tabular}{rcl}
\verb;x |> f(y); & $\Leftrightarrow$ & \verb;f(x, y);\\
\verb;x |> f(y) |> g(z); & $\Leftrightarrow$ & \verb;g(f(x, y), z);\\
\end{tabular}
\end{center}
\vfill

\begin{verbatim}
flights |>
  filter(dest == "IAH") |> 
  group_by(year, month, day) |> 
  summarize(
    arr_delay = mean(arr_delay, na.rm = TRUE)
  )
\end{verbatim}
\end{frame}

\bfr{Global options}
\fig{1}{rstudio-pipe-options.png}
\bi
\li Enable ``Use native pipe operator.''
\li This will enable you to produce the pipe with \verb|ctrl-shift-M|
\ei
\end{frame}

\bfr{filter}

\begin{verbatim}
flights |> 
  filter(dep_delay > 120)
#> # A tibble: 9,723 × 19
#>    year month   day dep_time sched_dep_time dep_delay arr_time sched_arr_time
#>   <int> <int> <int>    <int>          <int>     <dbl>    <int>          <int>
#> 1  2013     1     1      848           1835       853     1001           1950
#> 2  2013     1     1      957            733       144     1056            853
#> 3  2013     1     1     1114            900       134     1447           1222
#> 4  2013     1     1     1540           1338       122     2020           1825
#> 5  2013     1     1     1815           1325       290     2120           1542
#> 6  2013     1     1     1842           1422       260     1958           1535
#> # i 9,717 more rows
#> # i 11 more variables: arr_delay <dbl>, carrier <chr>, flight <int>, …
\end{verbatim}
\end{frame}

\bfr{arrange}

\begin{verbatim}
flights |> 
  arrange(year, month, day, dep_time)
#> # A tibble: 336,776 × 19
#>    year month   day dep_time sched_dep_time dep_delay arr_time sched_arr_time
#>   <int> <int> <int>    <int>          <int>     <dbl>    <int>          <int>
#> 1  2013     1     1      517            515         2      830            819
#> 2  2013     1     1      533            529         4      850            830
#> 3  2013     1     1      542            540         2      923            850
#> 4  2013     1     1      544            545        -1     1004           1022
#> 5  2013     1     1      554            600        -6      812            837
#> 6  2013     1     1      554            558        -4      740            728
#> # i 9,717 more rows
#> # i 11 more variables: arr_delay <dbl>, carrier <chr>, flight <int>, …
\end{verbatim}
\end{frame}

\bfr{distinct}
\begin{verbatim}
# Find all unique origin and destination pairs
flights |> 
  distinct(origin, dest)
#> # A tibble: 224 × 2
#>   origin dest 
#>   <chr>  <chr>
#> 1 EWR    IAH  
#> 2 LGA    IAH  
#> 3 JFK    MIA  
#> 4 JFK    BQN  
#> 5 LGA    ATL  
#> 6 EWR    ORD  
#> # i 218 more rows
\end{verbatim}
\end{frame}


\bfr{count}
\begin{verbatim}
flights |>
  count(origin, dest, sort = TRUE)
#> # A tibble: 224 × 3
#>   origin dest      n
#>   <chr>  <chr> <int>
#> 1 JFK    LAX   11262
#> 2 LGA    ATL   10263
#> 3 LGA    ORD    8857
#> 4 JFK    SFO    8204
#> 5 LGA    CLT    6168
#> 6 EWR    ORD    6100
#> # i 218 more rows
\end{verbatim}
\end{frame}

\bfr{Do exercise 6}
\end{frame}


\bfr{mutate}\scriptsize
\begin{verbatim}
flights |> 
  mutate(
    gain = dep_delay - arr_delay,
    speed = distance / air_time * 60,
    .before = 1
  )
#> # A tibble: 336,776 × 21
#>    gain speed  year month   day dep_time sched_dep_time dep_delay arr_time
#>   <dbl> <dbl> <int> <int> <int>    <int>          <int>     <dbl>    <int>
#> 1    -9  370.  2013     1     1      517            515         2      830
#> 2   -16  374.  2013     1     1      533            529         4      850
#> 3   -31  408.  2013     1     1      542            540         2      923
#> 4    17  517.  2013     1     1      544            545        -1     1004
#> 5    19  394.  2013     1     1      554            600        -6      812
#> 6   -16  288.  2013     1     1      554            558        -4      740
#> # i 336,770 more rows
#> # i 12 more variables: sched_arr_time <int>, arr_delay <dbl>, …
\end{verbatim}


\end{frame}

\bfr{select}\scriptsize
\begin{verbatim}
flights |> 
  select(year, month, day)

flights |> 
  select(year:day)

flights |> 
  select(!year:day)

flights |> 
  select(where(is.character))
\end{verbatim}

\bi
\li {\tt starts\_with("abc")}: matches names that begin with “abc”.
\li  {\tt ends\_with("xyz")}: matches names that end with “xyz”.
\li {\tt  contains("ijk")}: matches names that contain “ijk”.
\li  {\tt num\_range("x", 1:3)}: matches x1, x2 and x3.
\ei

\end{frame}


\bfr{rename}\scriptsize
\begin{verbatim}
flights |> 
  rename(tail_num = tailnum)
#> # A tibble: 336,776 × 19
#>    year month   day dep_time sched_dep_time dep_delay arr_time sched_arr_time
#>   <int> <int> <int>    <int>          <int>     <dbl>    <int>          <int>
#> 1  2013     1     1      517            515         2      830            819
#> 2  2013     1     1      533            529         4      850            830
#> 3  2013     1     1      542            540         2      923            850
#> 4  2013     1     1      544            545        -1     1004           1022
#> 5  2013     1     1      554            600        -6      812            837
#> 6  2013     1     1      554            558        -4      740            728
\end{verbatim}
\end{frame}

\bfr{relocate}\scriptsize
\begin{verbatim}
flights |> 
  relocate(time_hour, air_time)
#> # A tibble: 336,776 × 19
#>   time_hour           air_time  year month   day dep_time sched_dep_time
#>   <dttm>                 <dbl> <int> <int> <int>    <int>          <int>
#> 1 2013-01-01 05:00:00      227  2013     1     1      517            515
#> 2 2013-01-01 05:00:00      227  2013     1     1      533            529
#> 3 2013-01-01 05:00:00      160  2013     1     1      542            540
#> 4 2013-01-01 05:00:00      183  2013     1     1      544            545
#> 5 2013-01-01 06:00:00      116  2013     1     1      554            600
#> 6 2013-01-01 05:00:00      150  2013     1     1      554            558
\end{verbatim}

\end{frame}

\bfr{Do exercise 7}
\end{frame}

\bfr{The Pipe}\scriptsize
\begin{verbatim}
flights |> 
  filter(dest == "IAH") |> 
  mutate(speed = distance / air_time * 60) |> 
  select(year:day, dep_time, carrier, flight, speed) |> 
  arrange(desc(speed))
#> # A tibble: 7,198 × 7
#>    year month   day dep_time carrier flight speed
#>   <int> <int> <int>    <int> <chr>    <int> <dbl>
#> 1  2013     7     9      707 UA         226  522.
#> 2  2013     8    27     1850 UA        1128  521.
#> 3  2013     8    28      902 UA        1711  519.
#> 4  2013     8    28     2122 UA        1022  519.
#> 5  2013     6    11     1628 UA        1178  515.
#> 6  2013     8    27     1017 UA         333  515.
\end{verbatim}

\end{frame}

\bfr{The Pipe vs. function nesting}\scriptsize
\begin{verbatim}
arrange(
  select(
    mutate(
      filter(
        flights, 
        dest == "IAH"
      ),
      speed = distance / air_time * 60
    ),
    year:day, dep_time, carrier, flight, speed
  ),
  desc(speed)
)
\end{verbatim}
\end{frame}
\bfr{The Pipe vs. assignment to temporaries}\scriptsize
\begin{verbatim}
flights1 <- filter(flights, dest == "IAH")
flights2 <- mutate(flights1, speed = distance / air_time * 60)
flights3 <- select(flights2, year:day, dep_time, carrier, flight, speed)
arrange(flights3, desc(speed))
\end{verbatim}
\end{frame}


\bfr{\tt group\_by}
{\scriptsize
\begin{verbatim}
flights |> 
  group_by(month)
#> # A tibble: 336,776 × 19
#> # Groups:   month [12]
#>    year month   day dep_time sched_dep_time dep_delay arr_time sched_arr_time
#>   <int> <int> <int>    <int>          <int>     <dbl>    <int>          <int>
#> 1  2013     1     1      517            515         2      830            819
#> 2  2013     1     1      533            529         4      850            830
#> 3  2013     1     1      542            540         2      923            850
#> 4  2013     1     1      544            545        -1     1004           1022
#> 5  2013     1     1      554            600        -6      812            837
#> 6  2013     1     1      554            558        -4      740            728
\end{verbatim}
}
\bi
\li All subsequent operations will now work ``by month''
\ei
\end{frame}

\bfr{\tt summarize}
{\scriptsize
\begin{verbatim}
flights |> 
  group_by(month) |> 
  summarize(
    avg_delay = mean(dep_delay)
  )
#> # A tibble: 12 × 2
#>   month avg_delay
#>   <int>     <dbl>
#> 1     1        NA
#> 2     2        NA
#> 3     3        NA
#> 4     4        NA
#> 5     5        NA
#> 6     6        NA
\end{verbatim}
}
\bi
\li We forgot {\tt na.rm}
\ei
\end{frame}

\bfr{\tt summarize}
{\scriptsize
\begin{verbatim}
flights |> 
  group_by(month) |> 
  summarize(
    delay = mean(dep_delay, na.rm = TRUE)
  )
#> # A tibble: 12 × 2
#>   month delay
#>   <int> <dbl>
#> 1     1  10.0
#> 2     2  10.8
#> 3     3  13.2
#> 4     4  13.9
#> 5     5  13.0
#> 6     6  20.8
\end{verbatim}
}

\end{frame}
\bfr{{\tt summarize} with {\tt n}}
{\scriptsize
\begin{verbatim}
flights |> 
  group_by(month) |> 
  summarize(
    delay = mean(dep_delay, na.rm = TRUE), 
    n = n()
  )
#> # A tibble: 12 × 3
#>   month delay     n
#>   <int> <dbl> <int>
#> 1     1  10.0 27004
#> 2     2  10.8 24951
#> 3     3  13.2 28834
#> 4     4  13.9 28330
#> 5     5  13.0 28796
#> 6     6  20.8 28243
\end{verbatim}
}

\end{frame}


\bfr{{\tt slice}}
\bi
\li \verb$df |> slice_head(n = 1)$\\ takes the first row from each group.
\li \verb$df |> slice_tail(n = 1)$\\ takes the last row in each group.
\li \verb$df |> slice_min(x, n = 1)$ \\takes the row with the smallest value of column x.
\li \verb$df |> slice_max(x, n = 1)$\\ takes the row with the largest value of column x.
\li \verb$df |> slice_sample(n = 1)$ \\takes one random row.
\ei
\end{frame}

\bfr{\tt slice}\scriptsize
\begin{verbatim}
flights |> 
  group_by(dest) |> 
  slice_max(arr_delay, n = 1) |>
  relocate(dest)
#> # A tibble: 108 × 19
#> # Groups:   dest [105]
#>   dest   year month   day dep_time sched_dep_time dep_delay arr_time
#>   <chr> <int> <int> <int>    <int>          <int>     <dbl>    <int>
#> 1 ABQ    2013     7    22     2145           2007        98      132
#> 2 ACK    2013     7    23     1139            800       219     1250
#> 3 ALB    2013     1    25      123           2000       323      229
#> 4 ANC    2013     8    17     1740           1625        75     2042
#> 5 ATL    2013     7    22     2257            759       898      121
#> 6 AUS    2013     7    10     2056           1505       351     2347
\end{verbatim}
\end{frame}


\bfr{Groiuping by more than one variable}\scriptsize
\begin{verbatim}
daily <- flights |>  
  group_by(year, month, day)
daily
#> # A tibble: 336,776 × 19
#> # Groups:   year, month, day [365]
#>    year month   day dep_time sched_dep_time dep_delay arr_time sched_arr_time
#>   <int> <int> <int>    <int>          <int>     <dbl>    <int>          <int>
#> 1  2013     1     1      517            515         2      830            819
#> 2  2013     1     1      533            529         4      850            830
#> 3  2013     1     1      542            540         2      923            850
#> 4  2013     1     1      544            545        -1     1004           1022
#> 5  2013     1     1      554            600        -6      812            837
#> 6  2013     1     1      554            558        -4      740            728

\end{verbatim}
\end{frame}

\bfr{Case study}\scriptsize
\begin{verbatim}
batters <- Lahman::Batting |> 
  group_by(playerID) |> 
  summarize(
    performance = sum(H, na.rm = TRUE) / sum(AB, na.rm = TRUE),
    n = sum(AB, na.rm = TRUE)
  )
batters
#> # A tibble: 20,166 × 3
#>   playerID  performance     n
#>   <chr>           <dbl> <int>
#> 1 aardsda01      0          4
#> 2 aaronha01      0.305  12364
#> 3 aaronto01      0.229    944
#> 4 aasedo01       0          5
#> 5 abadan01       0.0952    21
#> 6 abadfe01       0.111      9
#> # i 20,160 more rows
\end{verbatim}
\end{frame}


\bfr{Case study}\scriptsize
\begin{verbatim}
batters |> 
  filter(n > 100) |> 
  ggplot(aes(x = n, y = performance)) +
  geom_point(alpha = 1 / 10) + 
  geom_smooth(se = FALSE)
\end{verbatim}
\fig{0.6}{unnamed-chunk-58-1.png}
\bi
\li The variation in performance is larger among players with fewer at-bats.
\li
There’s a positive correlation between skill (performance) and opportunities to hit the ball (n) because teams want to give their best batters the most opportunities to hit the ball.
\ei
\end{frame}

\bfr{Finding the best batters}
{\scriptsize
\begin{verbatim}
batters |> 
  arrange(desc(performance))
#> # A tibble: 20,166 × 3
#>   playerID  performance     n
#>   <chr>           <dbl> <int>
#> 1 abramge01           1     1
#> 2 alberan01           1     1
#> 3 banisje01           1     1
#> 4 bartocl01           1     1
#> 5 bassdo01            1     1
#> 6 birasst01           1     2
\end{verbatim}
}
\bi
\li \url{http://varianceexplained.org/r/empirical_bayes_baseball/}
\li \url{https://www.evanmiller.org/how-not-to-sort-by-average-rating.html}
\ei
\end{frame}

\bfr{pivot\_longer()}\scriptsize
\begin{verbatim}
billboard
#> # A tibble: 317 × 79
#>   artist       track               date.entered   wk1   wk2   wk3   wk4   wk5
#>   <chr>        <chr>               <date>       <dbl> <dbl> <dbl> <dbl> <dbl>
#> 1 2 Pac        Baby Don't Cry (Ke… 2000-02-26      87    82    72    77    87
#> 2 2Ge+her      The Hardest Part O… 2000-09-02      91    87    92    NA    NA
#> 3 3 Doors Down Kryptonite          2000-04-08      81    70    68    67    66
#> 4 3 Doors Down Loser               2000-10-21      76    76    72    69    67
#> 5 504 Boyz     Wobble Wobble       2000-04-15      57    34    25    17    17
#> 6 98^0         Give Me Just One N… 2000-08-19      51    39    34    26    26
#> # i 311 more rows
#> # i 71 more variables: wk6 <dbl>, wk7 <dbl>, wk8 <dbl>, wk9 <dbl>, …
\end{verbatim}
\end{frame}

\bfr{pivot\_longer()}\scriptsize
\begin{verbatim}
billboard |> 
  pivot_longer(
    cols = starts_with("wk"), 
    names_to = "week", 
    values_to = "rank"
  )
#> # A tibble: 24,092 × 5
#>    artist track                   date.entered week   rank
#>    <chr>  <chr>                   <date>       <chr> <dbl>
#>  1 2 Pac  Baby Don't Cry (Keep... 2000-02-26   wk1      87
#>  2 2 Pac  Baby Don't Cry (Keep... 2000-02-26   wk2      82
#>  3 2 Pac  Baby Don't Cry (Keep... 2000-02-26   wk3      72
#>  4 2 Pac  Baby Don't Cry (Keep... 2000-02-26   wk4      77
#>  5 2 Pac  Baby Don't Cry (Keep... 2000-02-26   wk5      87
#>  6 2 Pac  Baby Don't Cry (Keep... 2000-02-26   wk6      94
#>  7 2 Pac  Baby Don't Cry (Keep... 2000-02-26   wk7      99
#>  8 2 Pac  Baby Don't Cry (Keep... 2000-02-26   wk8      NA
#>  9 2 Pac  Baby Don't Cry (Keep... 2000-02-26   wk9      NA
#> 10 2 Pac  Baby Don't Cry (Keep... 2000-02-26   wk10     NA
#> # i 24,082 more rows
\end{verbatim}


\end{frame}

\bfr{pivot\_longer()}\scriptsize
\begin{verbatim}
billboard |> 
  pivot_longer(
    cols = starts_with("wk"), 
    names_to = "week", 
    values_to = "rank",
    values_drop_na = TRUE
  )
#> # A tibble: 5,307 × 5
#>   artist track                   date.entered week   rank
#>   <chr>  <chr>                   <date>       <chr> <dbl>
#> 1 2 Pac  Baby Don't Cry (Keep... 2000-02-26   wk1      87
#> 2 2 Pac  Baby Don't Cry (Keep... 2000-02-26   wk2      82
#> 3 2 Pac  Baby Don't Cry (Keep... 2000-02-26   wk3      72
#> 4 2 Pac  Baby Don't Cry (Keep... 2000-02-26   wk4      77
#> 5 2 Pac  Baby Don't Cry (Keep... 2000-02-26   wk5      87
#> 6 2 Pac  Baby Don't Cry (Keep... 2000-02-26   wk6      94
#> # i 5,301 more rows
\end{verbatim}


\end{frame}



\bfr{pivot\_longer()}\scriptsize
\begin{verbatim}
billboard_longer <- billboard |> 
  pivot_longer(
    cols = starts_with("wk"), 
    names_to = "week", 
    values_to = "rank",
    values_drop_na = TRUE
  ) |> 
  mutate(
    week = parse_number(week)
  )
billboard_longer
#> # A tibble: 5,307 × 5
#>   artist track                   date.entered  week  rank
#>   <chr>  <chr>                   <date>       <dbl> <dbl>
#> 1 2 Pac  Baby Don't Cry (Keep... 2000-02-26       1    87
#> 2 2 Pac  Baby Don't Cry (Keep... 2000-02-26       2    82
#> 3 2 Pac  Baby Don't Cry (Keep... 2000-02-26       3    72
#> 4 2 Pac  Baby Don't Cry (Keep... 2000-02-26       4    77
#> 5 2 Pac  Baby Don't Cry (Keep... 2000-02-26       5    87
#> 6 2 Pac  Baby Don't Cry (Keep... 2000-02-26       6    94
#> # i 5,301 more rows
\end{verbatim}


\end{frame}


\bfr{pivot\_longer()}\scriptsize
\begin{verbatim}
billboard_longer |> 
  ggplot(aes(x = week, y = rank, group = track)) + 
  geom_line(alpha = 0.25) + 
  scale_y_reverse()
\end{verbatim}

\fig{.9}{fig-billboard-ranks-1.png}

\end{frame}

\bfr{pivot\_longer()}\scriptsize
\begin{verbatim}
df <- tribble(
  ~id,  ~bp1, ~bp2,
   "A",  100,  120,
   "B",  140,  115,
   "C",  120,  125
)
df |> 
  pivot_longer(
    cols = bp1:bp2,
    names_to = "measurement",
    values_to = "value"
  )
#> # A tibble: 6 × 3
#>   id    measurement value
#>   <chr> <chr>       <dbl>
#> 1 A     bp1           100
#> 2 A     bp2           120
#> 3 B     bp1           140
#> 4 B     bp2           115
#> 5 C     bp1           120
#> 6 C     bp2           125
\end{verbatim}

\end{frame}

\bfr{pivot\_longer()}\scriptsize
\begin{verbatim}
df |> 
  pivot_longer(
    cols = bp1:bp2,
    names_to = "measurement",
    values_to = "value"
  )
\end{verbatim}

\fig{1}{variables.png}
\end{frame}


\bfr{pivot\_longer()}\scriptsize
\begin{verbatim}
df |> 
  pivot_longer(
    cols = bp1:bp2,
    names_to = "measurement",
    values_to = "value"
  )
\end{verbatim}
\fig{1}{column-names.png}
\end{frame}

\bfr{pivot\_longer()}\scriptsize
\begin{verbatim}
df |> 
  pivot_longer(
    cols = bp1:bp2,
    names_to = "measurement",
    values_to = "value"
  )
\end{verbatim}
\fig{1}{cell-values.png}
\end{frame}

\bfr{Many variables in column names}\scriptsize
\begin{verbatim}
who2
#> # A tibble: 7,240 × 58
#>   country      year sp_m_014 sp_m_1524 sp_m_2534 sp_m_3544 sp_m_4554
#>   <chr>       <dbl>    <dbl>     <dbl>     <dbl>     <dbl>     <dbl>
#> 1 Afghanistan  1980       NA        NA        NA        NA        NA
#> 2 Afghanistan  1981       NA        NA        NA        NA        NA
#> 3 Afghanistan  1982       NA        NA        NA        NA        NA
#> 4 Afghanistan  1983       NA        NA        NA        NA        NA
#> 5 Afghanistan  1984       NA        NA        NA        NA        NA
#> 6 Afghanistan  1985       NA        NA        NA        NA        NA
#> # i 7,234 more rows
#> # i 51 more variables: sp_m_5564 <dbl>, sp_m_65 <dbl>, sp_f_014 <dbl>, …
\end{verbatim}
\centerline{Tuberculosis diagnoses}
\vfill
\bi
\li {\tt sp/rel/ep}  describes the method used for the diagnosis
\li {\tt m/f} is the gender (coded as a binary variable in this dataset)
\li {\tt 014/1524/2534/3544/4554/5564/65} is the age range (014 represents 0-14, for example)
\ei
\end{frame}

\bfr{{\tt names\_to} {\tt names\_sep} {\tt values\_to}}\scriptsize
\begin{verbatim}
who2 |> 
  pivot_longer(
    cols = !(country:year),
    names_to = c("diagnosis", "gender", "age"), 
    names_sep = "_",
    values_to = "count"
  )
#> # A tibble: 405,440 × 6
#>   country      year diagnosis gender age   count
#>   <chr>       <dbl> <chr>     <chr>  <chr> <dbl>
#> 1 Afghanistan  1980 sp        m      014      NA
#> 2 Afghanistan  1980 sp        m      1524     NA
#> 3 Afghanistan  1980 sp        m      2534     NA
#> 4 Afghanistan  1980 sp        m      3544     NA
#> 5 Afghanistan  1980 sp        m      4554     NA
#> 6 Afghanistan  1980 sp        m      5564     NA
#> # i 405,434 more rows
\end{verbatim}
\end{frame}

\bfr{Data in column names}
\fig{1}{multiple-names.png}
\end{frame}


\bfr{Data and variable names in column header}\scriptsize
\begin{verbatim}
household
#> # A tibble: 5 × 5
#>   family dob_child1 dob_child2 name_child1 name_child2
#>    <int> <date>     <date>     <chr>       <chr>      
#> 1      1 1998-11-26 2000-01-29 Susan       Jose       
#> 2      2 1996-06-22 NA         Mark        <NA>       
#> 3      3 2002-07-11 2004-04-05 Sam         Seth       
#> 4      4 2004-10-10 2009-08-27 Craig       Khai       
#> 5      5 2000-12-05 2005-02-28 Parker      Gracie
\end{verbatim}
\end{frame}


\bfr{\tt .value}\scriptsize
\begin{verbatim}
household |> 
  pivot_longer(
    cols = !family, 
    names_to = c(".value", "child"), 
    names_sep = "_", 
    values_drop_na = TRUE
  )
#> # A tibble: 9 × 4
#>   family child  dob        name 
#>    <int> <chr>  <date>     <chr>
#> 1      1 child1 1998-11-26 Susan
#> 2      1 child2 2000-01-29 Jose 
#> 3      2 child1 1996-06-22 Mark 
#> 4      3 child1 2002-07-11 Sam  
#> 5      3 child2 2004-04-05 Seth 
#> 6      4 child1 2004-10-10 Craig
#> # i 3 more rows
\end{verbatim}
\end{frame}

\bfr{Data and variable names in header}
\fig{1}{names-and-values.png}
\end{frame}


\bfr{Widening data}\scriptsize
\begin{verbatim}
cms_patient_experience
#> # A tibble: 500 × 5
#>   org_pac_id org_nm                     measure_cd   measure_title   prf_rate
#>   <chr>      <chr>                      <chr>        <chr>              <dbl>
#> 1 0446157747 USC CARE MEDICAL GROUP INC CAHPS_GRP_1  CAHPS for MIPS…       63
#> 2 0446157747 USC CARE MEDICAL GROUP INC CAHPS_GRP_2  CAHPS for MIPS…       87
#> 3 0446157747 USC CARE MEDICAL GROUP INC CAHPS_GRP_3  CAHPS for MIPS…       86
#> 4 0446157747 USC CARE MEDICAL GROUP INC CAHPS_GRP_5  CAHPS for MIPS…       57
#> 5 0446157747 USC CARE MEDICAL GROUP INC CAHPS_GRP_8  CAHPS for MIPS…       85
#> 6 0446157747 USC CARE MEDICAL GROUP INC CAHPS_GRP_12 CAHPS for MIPS…       24
#> # i 494 more rows
\end{verbatim}
\bi
\li Each organization is spread over six rows.
\ei
\end{frame}

\bfr{Widening data}\scriptsize
\begin{verbatim}
cms_patient_experience |> 
  distinct(measure_cd, measure_title)
#> # A tibble: 6 × 2
#>   measure_cd   measure_title                                                 
#>   <chr>        <chr>                                                         
#> 1 CAHPS_GRP_1  CAHPS for MIPS SSM: Getting Timely Care, Appointments, and In…
#> 2 CAHPS_GRP_2  CAHPS for MIPS SSM: How Well Providers Communicate            
#> 3 CAHPS_GRP_3  CAHPS for MIPS SSM: Patient's Rating of Provider              
#> 4 CAHPS_GRP_5  CAHPS for MIPS SSM: Health Promotion and Education            
#> 5 CAHPS_GRP_8  CAHPS for MIPS SSM: Courteous and Helpful Office Staff        
#> 6 CAHPS_GRP_12 CAHPS for MIPS SSM: Stewardship of Patient Resources
\end{verbatim}
\bi
\li There are only six unique cd/title combinations
\ei
\end{frame}

\bfr{Widening data}\scriptsize
\begin{verbatim}
cms_patient_experience |> 
  pivot_wider(
    names_from = measure_cd,
    values_from = prf_rate
  )
#> # A tibble: 500 × 9
#>   org_pac_id org_nm                   measure_title   CAHPS_GRP_1 CAHPS_GRP_2
#>   <chr>      <chr>                    <chr>                 <dbl>       <dbl>
#> 1 0446157747 USC CARE MEDICAL GROUP … CAHPS for MIPS…          63          NA
#> 2 0446157747 USC CARE MEDICAL GROUP … CAHPS for MIPS…          NA          87
#> 3 0446157747 USC CARE MEDICAL GROUP … CAHPS for MIPS…          NA          NA
#> 4 0446157747 USC CARE MEDICAL GROUP … CAHPS for MIPS…          NA          NA
#> 5 0446157747 USC CARE MEDICAL GROUP … CAHPS for MIPS…          NA          NA
#> 6 0446157747 USC CARE MEDICAL GROUP … CAHPS for MIPS…          NA          NA
#> # i 494 more rows
#> # i 4 more variables: CAHPS_GRP_3 <dbl>, CAHPS_GRP_5 <dbl>, …
\end{verbatim}
\bi
\li We still have multiple rows for each organization.
\li We need to say which columns have values that
uniquely identify each row.
\ei
\end{frame}

\bfr{Widening data}\scriptsize
\begin{verbatim}
cms_patient_experience |> 
  pivot_wider(
    id_cols = starts_with("org"),
    names_from = measure_cd,
    values_from = prf_rate
  )
#> # A tibble: 95 × 8
#>   org_pac_id org_nm           CAHPS_GRP_1 CAHPS_GRP_2 CAHPS_GRP_3 CAHPS_GRP_5
#>   <chr>      <chr>                  <dbl>       <dbl>       <dbl>       <dbl>
#> 1 0446157747 USC CARE MEDICA…          63          87          86          57
#> 2 0446162697 ASSOCIATION OF …          59          85          83          63
#> 3 0547164295 BEAVER MEDICAL …          49          NA          75          44
#> 4 0749333730 CAPE PHYSICIANS…          67          84          85          65
#> 5 0840104360 ALLIANCE PHYSIC…          66          87          87          64
#> 6 0840109864 REX HOSPITAL INC          73          87          84          67
#> # i 89 more rows
#> # i 2 more variables: CAHPS_GRP_8 <dbl>, CAHPS_GRP_12 <dbl>
\end{verbatim}

\end{frame}

\bfr{\tt pivot\_wider}\scriptsize
\begin{verbatim}
df <- tribble(
  ~id, ~measurement, ~value,
  "A",        "bp1",    100,
  "B",        "bp1",    140,
  "B",        "bp2",    115, 
  "A",        "bp2",    120,
  "A",        "bp3",    105
)

df |> 
  pivot_wider(
    names_from = measurement,
    values_from = value
  )
#> # A tibble: 2 × 4
#>   id      bp1   bp2   bp3
#>   <chr> <dbl> <dbl> <dbl>
#> 1 A       100   120   105
#> 2 B       140   115    NA
\end{verbatim}

\end{frame}

\bfr{{\tt pivot\_wider} with duplicate rows}\scriptsize
\begin{verbatim}
df <- tribble(
  ~id, ~measurement, ~value,
  "A",        "bp1",    100,
  "A",        "bp1",    102,
  "A",        "bp2",    120,
  "B",        "bp1",    140, 
  "B",        "bp2",    115
)

df |>
  pivot_wider(
    names_from = measurement,
    values_from = value
  )
#> Warning: Values from `value` are not uniquely identified; output will contain
#> list-cols.
....
\end{verbatim}

\end{frame}

\end{document}

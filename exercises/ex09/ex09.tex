\documentclass[12pt]{article}
\usepackage[margin=1in]{geometry}
%\UseRawInputEncoding
\usepackage{amsmath,hyperref,graphicx}


\newcommand{\bi}{\begin{itemize}}
\newcommand{\ei}{\end{itemize}}
\newcommand{\li}{\item}
\newcommand{\fig}[2]{\centerline{\includegraphics[width=#1\textwidth]{#2}}}
\newcommand{\figg}[2]{\includegraphics[width=#1\textwidth]{#2}}

\newcommand{\arrow}{\ensuremath{\rightarrow}}
\newcommand{\lst}[1]{\lstinline{#1}}

\title{csci297b Exercise 9\\ggplot2
  }
\date{}
\sloppy

\begin{document}
\maketitle

\bi
\li
For this project, use the {\tt tidyverse} library, which includes the {\tt mpg} dataset.
\li
For this project, you will turn in a single R markdown file, called \verb|exercise09.Rmd|

\ei
\begin{enumerate}
\item
What \verb|geom| would you use to draw a line chart? A boxplot? A histogram? An area chart?
\item
Earlier in this chapter we used \verb|show.legend| without explaining it:
\begin{verbatim}
ggplot(mpg, aes(x = displ, y = hwy)) +
  geom_smooth(aes(color = drv), show.legend = FALSE)
\end{verbatim}
What does \verb|show.legend = FALSE| do here? What happens if you remove it? Why do you think we used it earlier?
\item
What does the \verb|se| argument to \verb|geom_smooth()| do?
\item
Recreate the R code necessary to generate the following graphs. Note that wherever a categorical variable is used in the plot, it’s drv.
\end{enumerate}
\figg{0.33}{unnamed-chunk-19-1.png}%
\figg{0.33}{unnamed-chunk-19-2.png}%
\figg{0.33}{unnamed-chunk-19-3.png}

\figg{0.33}{unnamed-chunk-19-4.png}%
\figg{0.33}{unnamed-chunk-19-5.png}%
\figg{0.33}{unnamed-chunk-19-6.png}
\end{document}
\documentclass[12pt]{article}
\usepackage[margin=1in]{geometry}
%\UseRawInputEncoding
\usepackage{amsmath,hyperref}


\newcommand{\bi}{\begin{itemize}}
\newcommand{\ei}{\end{itemize}}
\newcommand{\li}{\item}
\newcommand{\fig}[2]{\centerline{\includegraphics[width=#1\textwidth]{#2}}}
\newcommand{\figg}[2]{\includegraphics[width=#1\textwidth]{#2}}

\newcommand{\arrow}{\ensuremath{\rightarrow}}
\newcommand{\lst}[1]{\lstinline{#1}}

\title{csci297b Exercise 8\\ggplot2
  }
\date{}
\sloppy

\begin{document}
\maketitle

\bi
\li
For this project, use the {\tt tidyverse} library, which includes the {\tt mpg} dataset.
\li
For this project, you will turn in a single R markdown file, called \verb|exercise08.Rmd|

\ei
\begin{enumerate}

\item Create a scatterplot of \verb|hwy| vs. \verb|displ| where the points are pink filled in triangles.

\item Why did the following code not result in a plot with blue points?
\begin{verbatim}
ggplot(mpg) + 
  geom_point(aes(x = displ, y = hwy, color = "blue"))
\end{verbatim}
\item
What does the \verb|stroke| aesthetic do? What shapes does it work with? (Hint: use \verb|?geom_point|)
\item
What happens if you map an aesthetic to something other than a variable name, like 
\verb|aes(color = displ < 5)|? Note, you’ll also need to specify x and y.
\end{enumerate}
\end{document}
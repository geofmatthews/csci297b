\documentclass[12pt]{article}
\usepackage[margin=1in]{geometry}
%\UseRawInputEncoding
\usepackage{amsmath,hyperref,graphicx}


\newcommand{\bi}{\begin{itemize}}
\newcommand{\ei}{\end{itemize}}
\newcommand{\li}{\item}
\newcommand{\fig}[2]{\centerline{\includegraphics[width=#1\textwidth]{#2}}}
\newcommand{\figg}[2]{\includegraphics[width=#1\textwidth]{#2}}

\newcommand{\arrow}{\ensuremath{\rightarrow}}
\newcommand{\lst}[1]{\lstinline{#1}}

\title{csci297b Exercise 11\\stats
  }
\date{}
\sloppy

\begin{document}
\maketitle

\bi
\li
For this project, use the {\tt tidyverse} library, which includes the {\tt mpg} dataset.
\li
For this project, you will turn in a single R markdown file, called \verb|exercise10.Rmd|

\ei
\begin{enumerate}
\item

What does \verb|geom_col()| do? How is it different from \verb|geom_bar()|?
\item
What variables does \verb|stat_smooth()| compute? What arguments control its behavior?
\item
In our proportion bar chart, we need to set \verb|group = 1|. Why? In other words, what is the problem with these two graphs?
\begin{verbatim}
ggplot(diamonds, aes(x = cut, y = after_stat(prop))) + 
  geom_bar()
ggplot(diamonds, aes(x = cut, fill = color, y = after_stat(prop))) + 
  geom_bar()
\end{verbatim}
\end{enumerate}
\end{document}
\documentclass[12pt]{article}
\usepackage{alltt}
\usepackage{color}
\usepackage{hyperref}
\usepackage[margin=1in]{geometry}
\setlength{\textwidth}{7in}
\setlength{\textheight}{9in}
\setlength{\topmargin}{-0.5in}

\newcommand{\li}{\item}
\newcommand{\myitem}[1]{\item[#1]}

\begin{document}
\sloppy

\begin{center}
{\Large Final Report Specifications}
{\large Syllabus, CSCI297b, Scientific Visualization, Winter 2023}
\end{center}

Your final report will be an R project in its own folder.  All datasets you need for your
report should be in a \verb|data| folder within the project.  Export this project folder
and submit the resulting \verb|zip| file to canvas.  After uploading this
file to RStudio Workbench, I should be able to \verb|knit| the R markdown
document without any further resources.

The project should include an R markdown document
that is the final report, called \verb|final_report.Rmd|  It should consist of at
least the following sections:
\begin{description}
\item[Abstract]  An executive summary of your project, its goals, 
and what you accomplished.  Don't hide anything (teasers), but
outline quickly the major results of your research.
\item[Introduction] A discussion of the problem you have selected
to examine.  Why is this important?  What are our preconceptions?
\item[Data] A discussion of the data you are going to use.  Where was it obtained?
What processing (wrangling) did you have to do to make it tidy?
Did you have to merge more than one dataset to get the results you wanted?
\item[Graphs and discussion] At least five graphs of at least three types.
You should discuss the rationale for creating these graphs, 
the expected results and the unexpected results,
and why a particular graph type is good (or bad) for 
answering your questions with a visualization.
\item[Conclusion] A summing up of what you have learned,
both about the subject matter (climate, inequity, {\em etc.}),
and also about what you have learned about visualization
by doing this project.
\end{description}
\end{document}

\documentclass{beamer}
\usetheme{Singapore}
\usepackage{changepage}

%\usepackage{pstricks,pst-node,pst-tree}
\usepackage{amssymb,latexsym,dirtree}
\usepackage{tikz}
\usepackage{graphicx}
\usepackage{fancyvrb}
%\usepackage{hyperref}
\usepackage{fancybox}
\usepackage[listings]{tcolorbox}

\definecolor{codegreen}{rgb}{0,0.6,0}
\definecolor{codegray}{rgb}{0.5,0.5,0.5}
\definecolor{codepurple}{rgb}{0.58,0,0.82}
\definecolor{backcolour}{rgb}{0.95,0.95,0.92}

\lstdefinestyle{mystyle}{
    language=Python,
    backgroundcolor=\color{backcolour},   
    commentstyle=\color{codegreen},
    keywordstyle=\color{magenta},
    numberstyle=\tiny\color{codegray},
    stringstyle=\color{codepurple},
    basicstyle=\ttfamily\normalsize,
    breakatwhitespace=false,         
    breaklines=true,                 
    captionpos=b,                    
    keepspaces=true,                 
    numbers=left,                    
    numbersep=5pt,                  
    showspaces=false,                
    showstringspaces=false,
    showtabs=false,                  
    tabsize=2,
    escapechar=|,
    frame=single
}

\lstset{style=mystyle}


\newcommand{\lst}[1]{\lstinline{#1}}

\newcommand{\lsting}[1]{\begin{lstlisting}[basicstyle=#1]}
\newcommand{\lstend}{\end{lstlisting}}

\newcommand{\bi}{\begin{itemize}}
\newcommand{\li}{\item}
\newcommand{\ei}{\end{itemize}}
\newcommand{\Show}[1]{
\begin{center}
\shadowbox{\begin{minipage}{0.8\textwidth}
          #1
          \end{minipage}}
\end{center}
}
\newcommand{\arrow}{\ensuremath{\rightarrow}}

\newcommand{\uparr}{\ensuremath{\uparrow}}


\newcommand{\fig}[2]{\centerline{\includegraphics[width=#1\textwidth]{#2}}}

\newcommand{\bfr}[1]{\begin{frame}[fragile]\frametitle{{ #1 }}}
\newcommand{\efr}{\end{frame}}

\newcommand{\cola}{\begin{columns}\begin{column}{0.6\textwidth}}
\newcommand{\colb}{\end{column}\begin{column}{0.4\textwidth}}
\newcommand{\colc}{\end{column}\end{columns}}


\title{{https://intro2r.com/} Chapter 4}
\author{CSCI 297b, Spring 2023}

\begin{document}

\begin{frame}
\maketitle
\end{frame}

\bfr{Base, lattice, and ggplot2 graphics}

\bi
\li Base graphics:  easy, but good style takes work
\li Lattice graphics:  best with complex multi-dimensional data using panel plots
\li Grammar of graphics:  logical development, very good defaults
\ei
\end{frame}

\bfr{Plot panel in RStudio}
\fig{1}{bg_plots1}
\end{frame}

\bfr{Plot panel in RStudio, Zoom button}
\fig{1}{bg_plots2}
\end{frame}

\bfr{Plot panel in RStudio, save button}
\fig{1}{bg_plots3.png}
\end{frame}

\bfr{Scatterplots}
\lsting{\scriptsize}
flowers <- read.table(file = 'data/flower.txt', 
                        header = TRUE, sep = "\t", 
                        stringsAsFactors = TRUE)
plot(flowers$weight)
## or
## with(flowers, plot(weight)) 
\end{lstlisting}
\fig{.8}{plot1-1}
\end{frame}

\bfr{Scatterplots}
\lsting{\scriptsize}
plot(x = flowers$weight, y = flowers$shootarea)
## or
## plot(flowers$shootarea ~ flowers$weight)
\end{lstlisting}
\fig{.8}{plot4-1.png}
\end{frame}

\bfr{Scatterplots}
\lsting{\scriptsize}
my_x <- 1:10
my_y <- seq(from = 1, to = 20, by = 2)
par(mfrow = c(2, 2))
plot(my_x, my_y, type = "l")
plot(my_x, my_y, type = "b")
plot(my_x, my_y, type = "o")
plot(my_x, my_y, type = "c")
\end{lstlisting}
\fig{.6}{plot6-1.png}
\end{frame}

\bfr{Plot}
\bi
\li {\tt plot} has many options 
\li Can add more points, lines, text, {\em etc.}
\li {\tt plot} is a generic function:  it can change its behavior based
on what kind of object it is plotting
\ei
\end{frame}


\bfr{Histograms}
\lsting{\scriptsize}
hist(flowers$height)
\end{lstlisting}
\fig{.8}{plot7-1.png}
\end{frame}


\bfr{Histograms}
\lsting{\scriptsize}
brk <- seq(from = 0, to = 18, by = 1)
hist(flowers$height, breaks = brk, main = "petunia height")
\end{lstlisting}
\fig{.8}{plot8-1.png}
\end{frame}

\bfr{Histograms}
\lsting{\scriptsize}
brk <- seq(from = 0, to = 18, by = 1)
hist(flowers$height, breaks = brk, main = "petunia height",
      freq = FALSE)
\end{lstlisting}
\fig{.8}{plot9-1.png}
\end{frame}

\bfr{Histograms}
\lsting{\scriptsize}
dens <- density(flowers$height)
hist(flowers$height, breaks = brk, main = "petunia height",
      freq = FALSE)
lines(dens)
\end{lstlisting}
\fig{.8}{plot10-1.png}
\end{frame}

\end{document}

\documentclass{beamer}
\usetheme{Singapore}
\usepackage{changepage}
\usepackage[T1]{fontenc}

%\usepackage{pstricks,pst-node,pst-tree}
\usepackage{amssymb,latexsym,dirtree}
\usepackage{tikz}
\usepackage{graphicx}
\usepackage{fancyvrb}
%\usepackage{hyperref}
\usepackage{fancybox}
\usepackage[listings]{tcolorbox}

\definecolor{codegreen}{rgb}{0,0.6,0}
\definecolor{codegray}{rgb}{0.5,0.5,0.5}
\definecolor{codepurple}{rgb}{0.58,0,0.82}
\definecolor{backcolour}{rgb}{0.95,0.95,0.92}

\lstdefinestyle{mystyle}{
    language=Python,
    backgroundcolor=\color{backcolour},   
    commentstyle=\color{codegreen},
    keywordstyle=\color{magenta},
    numberstyle=\tiny\color{codegray},
    stringstyle=\color{codepurple},
    basicstyle=\ttfamily\normalsize,
    breakatwhitespace=false,         
    breaklines=true,                 
    captionpos=b,                    
    keepspaces=true,                 
    numbers=left,                    
    numbersep=5pt,                  
    showspaces=false,                
    showstringspaces=false,
    showtabs=false,                  
    tabsize=2,
    escapechar=|,
    frame=single
}

\lstset{style=mystyle}
\lstset{extendedchars=\true}



\newcommand{\bi}{\begin{itemize}}
\newcommand{\li}{\item}
\newcommand{\ei}{\end{itemize}}
\newcommand{\Show}[1]{
\begin{center}
\shadowbox{\begin{minipage}{0.8\textwidth}
          #1
          \end{minipage}}
\end{center}
}
\newcommand{\arrow}{\ensuremath{\rightarrow}}

\newcommand{\uparr}{\ensuremath{\uparrow}}


\newcommand{\fig}[2]{\centerline{\includegraphics[width=#1\textwidth]{#2}}}
\newcommand{\figg}[2]{\includegraphics[width=#1\textwidth]{#2}}

\newcommand{\bfr}[1]{\begin{frame}[fragile]\frametitle{{ #1 }}}
\newcommand{\efr}{\end{frame}}

\newcommand{\cola}[1]{\begin{columns}\begin{column}{#1\textwidth}}
\newcommand{\colb}[1]{\end{column}\begin{column}{#1\textwidth}}
\newcommand{\colc}{\end{column}\end{columns}}


\title{{https://intro2r.com/} Chapter 5}
\author{CSCI 297b, Spring 2023}

\begin{document}

\begin{frame}
\maketitle
\end{frame}

\bfr{R graphics}
\bi
\li  R provides easy, quick and dirty graphics for work and exploration
\li R also provides sophisticated, publication-ready graphics
\li R graphics are highly customizable, feeding our creative side
\li This enables better expression and communication
\li Opposite of ``click scatterplot button''
\ei
\end{frame}

\bfr{The grammar of graphics}
\bi
\li \url{https://link.springer.com/book/10.1007/0-387-28695-0}
\li \url{https://www.jstatsoft.org/article/view/v017b03/v17b03.pdf}
\li  \url{https://ggplot2-book.org/}
\ei
\end{frame}

\bfr{The grammar of graphics}
\bi
\li The user should be in control of all components (i.e. layers) and produce a tailor-made figure fit for their specific needs. 
\ei

\fig{0.5}{full_connected.png}

\end{frame}

\bfr{ggplot2}
\bi
\li In 2007 ggplot2 was released by Hadley Wickham. 
\li By 2017 the package had reportedly been downloaded 10 million times.
\li {\tt ggplot2} is not required for publication quality graphics.
\li Base graphics can do the job, it's just easier with {\tt ggplot2}
\ei

\end{frame}

\bfr{A final figure}
\fig{.9}{final-figure-start-1.png}
\bi
\li We will develop this figure step-by-step.
\ei
\end{frame}

\bfr{{\tt ggplot2} library}
\begin{verbatim}
install.packages("ggplot2")
library(ggplot2)
\end{verbatim}
\bi
\li Installing the package is not necessary on RStudio Workbench
\ei
\end{frame}

\bfr{The base {\tt ggplot}}
\begin{verbatim}
ggplot()
\end{verbatim}
\fig{.8}{pure-ggplot-1.png}
\end{frame}



\bfr{A reminder of the flower data set}
\begin{verbatim}
flower <- read.table("data/flower.csv", 
                           stringsAsFactors = TRUE, 
                           header = TRUE, sep = ",")
str(flower)
## 'data.frame':    96 obs. of  8 variables:
##  $ treat    : Factor w/ 2 levels "notip","tip": 2 2 2 2 2 2 2 2 2 2 ...
##  $ nitrogen : Factor w/ 3 levels "high","low","medium": 3 3 3 3 3 3 3 3 3 3 ...
##  $ block    : int  1 1 1 1 1 1 1 1 2 2 ...
##  $ height   : num  7.5 10.7 11.2 10.4 10.4 9.8 6.9 9.4 10.4 12.3 ...
##  $ weight   : num  7.62 12.14 12.76 8.78 13.58 ...
##  $ leafarea : num  11.7 14.1 7.1 11.9 14.5 12.2 13.2 14 10.5 16.1 ...
##  $ shootarea: num  31.9 46 66.7 20.3 26.9 72.7 43.1 28.5 57.8 36.9 ...
##  $ flowers  : int  1 10 10 1 4 9 7 6 5 8 ...
\end{verbatim}
\end{frame}

\bfr{Getting started}
\bi
\li We want {\tt shootarea}
on the $y$ axis and {\tt weight} on the $x$ axis.
\li To do this we specify a {\tt mapping} which is 
an {\tt aesthetic}.
\ei
\scriptsize
\begin{verbatim}
# Including aesthetics for x and y axes as well as specifying the dataset
ggplot(mapping = aes(x = weight, y = shootarea), data = flower)
\end{verbatim}


\figg{.6}{ggplot-with-xy-axis-1.png}\hfill\figg{.2}{data_mapping.png}

\end{frame}

\bfr{To see something we need geometry layers}
\scriptsize
\begin{verbatim}
ggplot(aes(x = weight, y = shootarea), data = flower) +
        geom_point()     # Adding a geom to display data as point data
\end{verbatim}

\fig{.8}{ggplot-with-geom_point-1.png}

\end{frame}

\bfr{ggplot is like painting in layers}
{
\scriptsize
\begin{verbatim}
ggplot(aes(x = weight, y = shootarea), data = flower) +
        geom_point()     # Adding a geom to display data as point data
\end{verbatim}
}
\bi
\li There are three essential layers


\figg{0.15}{data}\hfill\figg{0.15}{mapping}\hfill\figg{0.15}{geoms}

\li Other layers are optional, defaults handle most things.
\ei
\end{frame}

\bfr{Add line geometry layer}
\begin{verbatim}
ggplot(aes(x = weight, y = shootarea), data = flower) +
    geom_point() +
    geom_line()     # Adding geom_line
\end{verbatim}

\fig{0.8}{geom_line-1.png}
\end{frame}

\bfr{Add smooth geometry layer}
\begin{verbatim}
ggplot(aes(x = weight, y = shootarea), data = flower) +
    geom_point() +
    geom_smooth()     # Adding geom_line
\end{verbatim}

\fig{0.8}{geom_smooth-1.png}
\end{frame}

\bfr{LOESS default}
\fig{0.8}{geom_smooth-1.png}
\scriptsize
\bi
\li locally estimated scatterplot smoothing
\li We want a simple linear fit, from a linear model
\li Also, remove the confidence interval
\ei
\end{frame}

\bfr{Add linear model line layer}
\scriptsize
\begin{verbatim}
ggplot(aes(x = weight, y = shootarea), data = flower) +
    geom_point() +
    geom_smooth(method = "lm", se = FALSE)    # method and se 
\end{verbatim}

\fig{0.8}{geom_smooth-lm-1.png}
\end{frame}


\bfr{Add linear models for each level of nitrogen}
\scriptsize
\begin{verbatim}
ggplot(aes(x = weight, y = shootarea), data = flower) +
  geom_point() +
  # Including colour argument in aes()
  geom_smooth(aes(color = nitrogen), method = "lm", se = FALSE)
\end{verbatim}

\fig{0.8}{geom_smooth-colour-lm-1.png}
\end{frame}

\bfr{Where to put information}
{
\scriptsize
\begin{verbatim}
ggplot(aes(x = weight, y = shootarea), data = flower) +
  geom_point() +
  # Including colour argument in aes()
  geom_smooth(aes(color = nitrogen), method = "lm", se = FALSE)
  
ggplot() +
  # Moved aes() and data into geoms
  geom_point(aes(x = weight, y = shootarea), data = flower) +
  geom_smooth(aes(x = weight, y = shootarea, colour = nitrogen), 
                        data = flower, method = "lm", se = FALSE)
\end{verbatim}
}
\bi
\li When we include information such as {\tt data =} and {\tt aes()} in {\tt ggplot()}
 we are setting those as the default, universal values which all subsequent geoms use. 
\li Whereas if we were to include that information within a geom, only that geom would use that specific information. 
\ei

\end{frame}



\bfr{Moving {\tt color} into {\tt ggplot} colors points, too}
\scriptsize
\begin{verbatim}
# Moved colour = nitrogen into the universal ggplot()
ggplot(aes(x = weight, y = shootarea, color = nitrogen), data = flower) +
  geom_point() +
  geom_smooth(method = "lm", se = FALSE)
\end{verbatim}

\fig{0.8}{universal-colours-1.png}
\end{frame}

\bfr{Wrapping grids}

\fig{1}{final-figure-start-1.png}
\end{frame}

\bfr{Facets}
\scriptsize
\begin{verbatim}
ggplot(aes(x = weight, y = shootarea, colour = nitrogen), data = flower) +
  geom_point() +
  geom_smooth(method = "lm", se = FALSE) +
    # Splitting the single figure into multiple depending on treatment
  facet_wrap(~ treat)
\end{verbatim}
\fig{.8}{facet_wrap-1.png}
\end{frame}

\bfr{Facets}
\scriptsize
\begin{verbatim}
ggplot(aes(x = weight, y = shootarea, colour = nitrogen), data = flower) +
  geom_point() +
  geom_smooth(method = "lm", se = FALSE) +
    # Adding "block" to formula
  facet_wrap(~ treat + block)
\end{verbatim}
\fig{.8}{facet_wrap-formula-1.png}
\end{frame}

\bfr{Facets}
\scriptsize
\begin{verbatim}
ggplot(aes(x = weight, y = shootarea, colour = nitrogen), data = flower) +
  geom_point() +
  geom_smooth(method = "lm", se = FALSE) +
    # Changing to facet_grid
  facet_grid(~ treat + block)
\end{verbatim}
\fig{.8}{facet_grid-1.png}
\end{frame}

\bfr{Facets}
\scriptsize
\begin{verbatim}
ggplot(aes(x = weight, y = shootarea, colour = nitrogen), data = flower) +
  geom_point() +
  geom_smooth(method = "lm", se = FALSE) +
    # Rearranging formula, block in relation to treatment
  facet_grid(block ~ treat)
\end{verbatim}
\fig{.8}{facet-grid-formula-1.png}
\end{frame}

\bfr{Multiple figures using {\tt patchwork}}

\begin{verbatim}
library(patchwork)
\end{verbatim}

\bi
\li Arrange figures  above and below each other with $/$
\li Arrange figures side by side with $+$ or $\vert$
\ei
\end{frame}

\bfr{Assign plots to objects}

\begin{verbatim}
rbook_figure <- ggplot(aes(x = weight, 
                     y = shootarea,
                     colour = nitrogen), data = flower) +
  geom_point() +
  geom_smooth(method = "lm", se = FALSE) +
  facet_grid(block ~ treat)
\end{verbatim}
\bi
\li The figure we just constructed
\li Nothing is plotted if we assign it to an object
\li To see the figure, enter its name at the console
\ei
\end{frame}

\bfr{Arrange named figures with {\tt patchwork}}
\begin{verbatim}
rbook_figure / final_figure
\end{verbatim}
\fig{.9}{side-by-side-patchwork-1.png}
\end{frame}

\bfr{Caption individual figures with {\tt patchwork}}
\scriptsize
\begin{verbatim}
nested_compare <- rbook_figure / final_figure
nested_compare +
  plot_annotation(tag_levels = "A", tag_suffix = ")")
\end{verbatim}
\fig{.9}{labelled-nested-1.png}
\end{frame}

\bfr{Changing point size}
\scriptsize
\begin{verbatim}
ggplot(aes(x = weight, y = shootarea, colour = nitrogen), data = flower) +
    geom_point(size = 2) +
    geom_smooth(method = "lm", se = FALSE) +
    facet_grid(block ~ treat)
\end{verbatim}
\fig{.8}{change-size-1.png}
\end{frame}


\bfr{Changing point shape}
\scriptsize
\begin{verbatim}
ggplot(aes(x = weight, y = shootarea, colour = nitrogen), data = flower) +
    geom_point(aes(shape = nitrogen), size = 2) +
    geom_smooth(method = "lm", se = FALSE) +
    facet_grid(block ~ treat)
\end{verbatim}
\fig{.8}{change-shape-1.png}
\end{frame}


\bfr{Changing point transparency}
\scriptsize
\begin{verbatim}
ggplot(aes(x = weight, y = shootarea, colour = nitrogen), data = flower) +
    geom_point(aes(shape = nitrogen), size = 2, alpha = 0.6) +
    geom_smooth(method = "lm", se = FALSE) +
    facet_grid(block ~ treat)
\end{verbatim}
\fig{.8}{change-alpha-1.png}
\end{frame}


\bfr{Changing axis labels}
\scriptsize
\begin{verbatim}
ggplot(aes(x = weight, y = shootarea, colour = nitrogen), data = flower) +
    geom_point(aes(shape = nitrogen), size = 2, alpha = 0.6) +
    geom_smooth(method = "lm", se = FALSE) +
    facet_grid(block ~ treat) +
    xlab("Weight of flower (g)") +
    ylab("Area of shoot (cm^2)")
\end{verbatim}
\fig{.7}{axis-labels-1.png}
\end{frame}


\bfr{Special character interpretation with {\tt bquote}}
\scriptsize
\begin{verbatim}
ggplot(aes(x = weight, y = shootarea, colour = nitrogen), data = flower) +
    geom_point(aes(shape = nitrogen), size = 2, alpha = 0.6) +
    geom_smooth(method = "lm", se = FALSE) +
    facet_grid(block ~ treat) +
    xlab("Weight of flower (g)") +
    ylab(bquote("Area of shoot"~(cm^2)))
\end{verbatim}
\fig{.7}{bquote-1.png}
\end{frame}


\bfr{Fix some labels}
\scriptsize
\begin{verbatim}
ggplot(aes(x = weight, y = shootarea, colour = nitrogen), data = flower) +
    geom_point(aes(shape = nitrogen), size = 2, alpha = 0.6) +
    geom_smooth(method = "lm", se = FALSE) +
    facet_grid(block ~ treat) +
    xlab("Weight of flower (g)") +
    ylab(bquote("Area of shoot"~(cm^2))) +
    labs(shape = "Nitrogen Concentration", color = "Nitrogen Concentration", 
             caption = "Regression assumptions are unvalidated")
\end{verbatim}
\fig{.6}{extra-labels-1.png}
\end{frame}

\bfr{Using {\tt \textbackslash n} for newlines}
\scriptsize
\begin{verbatim}
ggplot(aes(x = weight, y = shootarea, color = nitrogen), data = flower) +
    geom_point(aes(shape = nitrogen), size = 2, alpha = 0.6) +
    geom_smooth(method = "lm", se = FALSE) +
    facet_grid(block ~ treat) +
    xlab("Weight of flower (g)") +
    ylab(bquote("Area of shoot"~(cm^2))) +
    labs(shape = "Nitrogen\nConcentration", color = "Nitrogen\nConcentration", 
             caption = "Regression assumptions are unvalidated")
\end{verbatim}
\fig{.6}{compact-labels-1.png}
\end{frame}

\bfr{Themes}
\scriptsize
\begin{verbatim}
classic <- rbook_figure  + theme_classic()
bw <- rbook_figure       + theme_bw()
minimal <- rbook_figure  + theme_minimal()
light <- rbook_figure    + theme_light()
(classic | bw) / (minimal | light)
\end{verbatim}

\fig{.8}{testing-themes-1.png}
\end{frame}

\bfr{More themes}
\bi
\li Package {\tt ggthemes}
\li Write your own theme (see text)
\ei
\end{frame}

\bfr{Add horizontal line}
\begin{verbatim}
rbook_figure + 
    geom_hline(aes(yintercept = mean(shootarea)), 
          linewidth = 0.5, colour = "black", linetype = 3) 
\end{verbatim}
\fig{.8}{geom_hline-1.png}

\end{frame}

\bfr{Adding a second {\tt geom\_smooth}}
\scriptsize
\begin{verbatim}
ggplot(aes(x = weight, y = shootarea, color = nitrogen), data = flower) +
    geom_point(aes(shape = nitrogen), size = 2, alpha = 0.6) +
    geom_smooth(method = "lm", se = FALSE) +
    # Adding a SECOND geom_smooth :O
    geom_smooth(method = "lm", se = FALSE, linetype = 2, alpha = 0.6, 
                         color = "black")  # overrides the default
    ....
\end{verbatim}
   

\fig{.7}{geom_smooth2-1.png}

\end{frame}
\bfr{Changing colors and scale labels manually}
\scriptsize
\begin{verbatim}
ggplot(aes(x = weight, y = shootarea, colour = nitrogen), data = flower) +
    ...
    # Setting colour and associated labels
    scale_colour_manual(values = c("#5C1AAE", "#AE5C1A", "#1AAE5C"),
       labels = c("High", "Medium", "Low")) +
    # Setting shape and associated labels
    scale_shape_manual(values = c(15,17,19),
       labels = c("High", "Medium", "Low"))
\end{verbatim}
   

\fig{.7}{scale-manual-1.png}

\end{frame}
\bfr{Labels are incorrect}

\fig{.7}{scale-manual-1.png}

\bi
\li High, Medium and Low were arbitrary
\li
Fix with:
\ei
\scriptsize
\begin{verbatim}
flower$nitrogen <- factor(flower$nitrogen, levels = c("high", "medium", "low"))
\end{verbatim}

\end{frame}

\bfr{Saving ggplots}
\begin{verbatim}
ggsave(filename = "areashoot_weight_facet.pdf", 
       plot = rbook_figure, 
       device = "pdf", 
       path = "output", 
       width = 250, 
       height = 150, 
       units = "mm", 
       dpi = 500, 
       limitsize = TRUE)
\end{verbatim}
\bi
\li If {\tt plot} is not specified, last thing plotted is saved.
\ei
\end{frame}

\end{document}
